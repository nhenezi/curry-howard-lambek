\documentclass[ a4paper, 12pt]{report}
%
\usepackage[text={15cm,22cm},centering]{geometry} % geometrija stranice
\usepackage[utf8x]{inputenc}
\usepackage[croatian]{babel}
\usepackage{amssymb}
\usepackage{graphicx} % paket za umetanje slika
%\usepackage{psfrag} % paket za pisanje po slikama
\usepackage{ccaption} % paket za kontroliranje "caption"-a za table, figure, ...
\usepackage{amsmath}
\usepackage{amsthm}
\usepackage{amsfonts}
\usepackage{bussproofs}
\usepackage{color}
\usepackage{longtable}
\usepackage{tabu}
\usepackage{tikz}
\usepackage{extpfeil}
\usepackage{hyperref}
\usepackage{listings}
\lstnewenvironment{mcode}{\lstset{language=Haskell,basicstyle=\small}}{}
\newextarrow{\xbigtoto}{{20}{20}{20}{20}}
{\bigRelbar\bigRelbar{\bigtwoarrowsleft\rightarrow\rightarrow}}

\nonstopmode
\definecolor{codegray}{gray}{0.9}
\newcommand{\category}[1]{\textbf{\emph{#1}}}
\newcommand{\codei}[1]{
  {\lstinline[basicstyle=\ttfamily]{#1}}
}
\newcommand{\code}[1]{
  \begin{align*}
    \texttt{#1}
  \end{align*}
  }
\theoremstyle{definition}

%
\makeatletter
\def\@makechapterhead#1{%
  %\vspace*{30\p@}%
  \noindent
  \begin{tabular*}{1\textwidth}{r}
  \hline
  \end{tabular*}
  {\parindent \z@ \raggedright \normalfont
  \begin{center}
  \ifnum \c@secnumdepth >\m@ne
      \huge\bfseries \thechapter \space \space
  \fi
  \interlinepenalty\@M
  \huge \bfseries #1\par\nobreak
  \end{center}
  %\vskip 20\p@
  }
  \noindent
  \begin{tabular*}{1\textwidth}{r}
  \hline
  \hline
  \end{tabular*}
  \vskip 20\p@
}
\def\section{\@startsection {section}{1}{\z@}%
                                   {-3.5ex \@plus -1ex \@minus -.2ex}%
                                   {2.3ex \@plus.2ex}%
                                   {\normalfont\large\bfseries}}
\makeatother
%
\newcounter{Primjer}
\numberwithin{Primjer}{chapter}
\setcounter{Primjer}{1}
\newcommand{\primjer}[1]{~\vskip 10pt
  {\footnotesize{
      \noindent
          {\textsl{Primjer \addtocounter{Primjer}{1} \thechapter.\arabic{Primjer} }}
          \vskip 5pt
          #1
          \vskip 10pt
  }}
}
%
\newtheorem{definition}{Definicija}[chapter]
\newtheorem{koloral}{Koloral}[chapter]
\newtheorem{teorem}{Teorem}[chapter]
\newtheorem{lema}{Lema}[chapter]

%
\renewcommand{\theequation}{\arabic{chapter}.\arabic{section}.\arabic{equation}}
%
\flushbottom
\addtolength{\voffset}{-0.5cm}
%\renewcommand{\baselinestretch}{2}
%  odkomentirajte ako zelite dvostruki prored
%
\captiondelim{}
%  ako necete imati, kod slika ili tablica, captione
%  npr: samo "Slika 1" a ne "Slika 1: Nesto"
%  inace zakomentirajte
%
\newcommand{\bi}[1]{\textbf{\em{#1}}}
% naredba "\bi{nesto}" tekst nesto stavlja u bold i italic
%
%\includeonly{}
%  odkomentirajte ako zelite da se kompajlira samo Sazetak + Poglavlja
%  u \includeonly{} mozete dodati sve sto se includa u ovoj datoteci
%
%%%%%%%%%%%%%%%%%%%%%%%%%%%%%%%%%%%%%%%
% Informacije potrebne za naslovnicu.
% Mogu se koristiti i izvan naslovnice.
%
\newcommand{\Fakultet}{
  Sveu\v{c}ili\v{s}te u Zagrebu \\[5mm]
  PMF - Matemati\v{c}ki odjel
}
\newcommand{\ImePrezimeAutora}{Nikola Henezi}
\newcommand{\NaslovRada}{Curry-Howard-Lambekov izomorfizam}
\newcommand{\ImePrezimeMentora}{Mladen Vukovi\'c}
\newcommand{\MjestoDatum}{mjesto, mjesec godina}
%
%%%%%%%%%%%%%%%%%%%%%%%%%%%%%%%%%%%%%%%%%%%%%%%%%%%%%%%%%%%%%%%%%%%%%%%%%%%%%%%%
%%%%%%%%%%%%%%%%%%%%%%%%%%%%%%%%%%%%%%%%%%%%%%%%%%%%%%%%%%%%%%%%%%%%%%%%%%%%%%%%
% Pocetak dokumenta
%
\begin{document}
%
%%%%%%%%%%%%%%%%%%%%%%%%%%%%%%%%%%%%
% Naslovnica diplomskog rada
%%%%%%%%%%%%%%%%%%%%%%%%%%%%%%%%%%%%
%
\thispagestyle{empty}
%
\begin{center}
  \Large{
    \Fakultet
  }
\end{center}
\vskip 150pt
\begin{center}
  \LARGE{
    \lineskip .75em
    \begin{tabular}[t]{c}
      \ImePrezimeAutora
    \end{tabular}
    \par
  }
  \vskip 3em
  \huge{
    \NaslovRada
    \par
  }
  \vskip 3em
  \Large{
    Diplomski rad
  }
\end{center}
\par
%
\vskip 125pt
\vfill
%
\begin{center}
  \large{
    \MjestoDatum
    \par
  }
\end{center}
%
\newpage
%
\thispagestyle{empty}
%
\begin{center}
  \Large{
    \Fakultet
  }
\end{center}
\vskip 150pt
\begin{center}
  \LARGE{
    \lineskip .75em
    \begin{tabular}[t]{c}
      \ImePrezimeAutora
    \end{tabular}
    \par
  }
  \vskip 3em
  \huge{
    \NaslovRada
    \par
  }
  \vskip 3em
  \Large{
    Diplomski rad
  }
  \vskip 70pt
  \Large{ Voditelj rada:}
  \\
  \ImePrezimeMentora
\end{center}
\par
\vfill
\vskip 40pt
%
\begin{center}
  \large{
    \MjestoDatum
    \par
  }
\end{center}
%

%%%%%%%%%%%%%%%%%%%%%%%%%%%%%%%%%%%%%%%%%%
% Povjerenstvo i ocjene, sluzbeni oblik
%%%%%%%%%%%%%%%%%%%%%%%%%%%%%%%%%%%%%%%%%%
%
\thispagestyle{empty}
%
Ovaj diplomski rad obranjen je dana
\begin{tabular}{p{5.0 cm}}
\\
\hline
\end{tabular}
pred nastavni\v{c}kim povjerenstvom
u sastavu:
%
\vskip 30pt
\begin{enumerate}
\item{
\begin{tabular}{p{8.0 cm}}

\\
\hline
\end{tabular}
, predsjednik
}
\vskip 20pt
\item{
\begin{tabular}{p{8.0 cm}}
\\
\hline
\end{tabular}
, \v{c}lan
}
\vskip 20pt
\item{
\begin{tabular}{p{8.0 cm}}
\\
\hline
\end{tabular}
, \v{c}lan
}
\end{enumerate}
\vskip 25pt
%
Povjerenstvo je rad ocijenilo ocjenom
\begin{tabular}{p{5.0 cm}}
\\
\hline
\end{tabular}.
\vskip 25pt
%
\noindent
Potpisi \v{c}lanova povjerenstva:
\vskip 30pt
%
\begin{enumerate}
\item{
\begin{tabular}{p{7.0 cm}}
\\
\hline
\end{tabular}
} %\item
\vskip 40pt
\item{
\begin{tabular}{p{7.0 cm}}
\\
\hline
\end{tabular}
} %\item
\vskip 40pt
\item{
\begin{tabular}{p{7.0 cm}}
\\
\hline
\end{tabular}
} %\item
\end{enumerate}


%
%%%%%%%%%%%%%%%%%%%%%%%%%%%%%%%%%%%%%%%%%%%%%
% Sadrzaj
%
\newpage
\pagenumbering{roman} \setcounter{page}{1}
\tableofcontents
%
%%%%%%%%%%%%%%%%%%%%%%%%%%%%%%%%%%%%%%%%%%%%%
%%%%%%%%%%%%%%%%%%%%%%%%%%%%%%%%%%%%%%%%%%%%%
%
%%%%%%%%%%%%%%%%%
% Uvod
%%%%%%%%%%%%%%%%%
%
\chapter*{Uvod}
\addcontentsline{toc}{chapter}{Uvod}
%
%%%%%%%%%%%%%%%%%%%%%%%%%%%%%%%%%%%%%%%%%%%
% 



%
\pagenumbering{arabic} \setcounter{page}{1}
%
%%%%%%%%%%%%%%%%%%%%%%%%%%%%%%%%%%%%%%%%%%%%%
%
\chapter{Teorija Kategorija}
%%%%%%%%%%%%%%%%%%%%%%%%%%%%%%%%%%%%%%%%%%%%%
%
  U ovom poglavlju definiramo kategorije i detaljno raspisujemo nekoliko
  primjera kategorija. 
  U točki "Dijagram" uvodimo koncept dijagrama i tehniku "praćenja
  dijagrama" koja olakšava neka razmatranja u teoriji kategorija.
  Iduće točke posvećujemo  "strelicama" i "objektima". Pritom navodimo nekoliko
  posebnih slučajava, proučavamo njihove strukture i izdvajamo nekoliko nama
  zanimljivih slučajeva.
  U posljednjoj točki ovog poglavlja bavimo se preslikavanjem između kategorija i
  demonstriramo praktičnu primjenu kategorije teorija u funkcijskom programskom jeziku Haskell.

  %%%%%%%%%%%%%%%%%%%%%%%%%%%
  %% Definicija Kategorije %%
  %%%%%%%%%%%%%%%%%%%%%%%%%%%
  \section{Kategorije}
  Teorija kategorije proučava "objekte" i "preslikavanja" između
  njih. Objekti i preslikavanja su primitivni objekti u teoriji kategorija i
  njih ne definiramo. Objekti ne moraju biti kolekcije elemenata i
  preslikavanja ne moraju biti funkcije na skupovima.
  U ovom poglavlju definiramo kategorije i detaljno raspisujemo nekoliko
  primjera.
  \begin{definition}\ \\
  
	\noindent Kategorija $G$ sastoji se od:
    \begin{itemize}
      \item klase objekata $Obj$
      \item klase strelica $Arw$
      \item preslikavanja $Arw \xrightarrow{source} Obj$
      \item preslikavanja $Arw \xrightarrow{target} Obj$
      \item preslikavanja (identitete) $Obj \xrightarrow{id} Arw$ takvog da
      za $B \xrightarrow{id} id_B$ vrijedi:
        \begin{equation*}
          target(id_B) = source(id_B) = B
        \end{equation*}
      \item preslikavanja (kompozicije) $Arw \times Arw \xrightarrow{\circ}
      Arw$ takvog da za \\ $(f, g) \xrightarrow{\circ} f \circ g$ vrijedi:
        \begin{align*}
          source(f) &= target(g) \\
          source(f \circ g) &= source(g) \\
          target(f \circ g) &= target(f)
        \end{align*}
    \end{itemize}
    
    Preslikavanje identiteta ($id$) i kompozicija ($\circ$) moraju
    zadovoljavati:
    \begin{itemize}
      \item svojstvo identiteta: za svaku strelicu $A \xrightarrow{f} B$ vrijedi:
        \begin{align*}
          id_B \circ f = f = f \circ id_A
        \end{align*}
      \item svojstvo asocijativnost: za sve strelice $A \xrightarrow{f} B
      \xrightarrow{g} C \xrightarrow{h} D$ vrijedi:
        \begin{align*} \label{def:kat_assoc}
          (h \circ g) \circ f = h \circ (g \circ f)
        \end{align*}
    \end{itemize}
  \end{definition}
  Kada želimo posebno naglasiti kojoj kategoriji pripadaju strelice i objekti
  onda ćemo umjesto $Obj$ i $Arw$ pisati $Obj_G$ i $Arw_G$.
  Za $A, B \in Obj_G$ skup svih strelica sa $A$ u $B$ označavamo sa
  \category{G}$[A, B]$.
  Sada ćemo detaljno raspisati nekoliko primjera kategorija.\\

  %%%%%%%%%%%%%%%%%%%%
  %% Mon Kategorija %%
  %%%%%%%%%%%%%%%%%%%%
  \begin{example}\ \\
   
    \noindent \textbf{Monoid} je uređena trojka $(M, \cdot_M, 1_M)$ gdje je $M$ skup, $1_M
    \in M$, $\cdot_M$ binarna operacija na $M$ za koju vrijedi da za svaki $a, b, c \in M$:
    \begin{equation*}
      (a \cdot_M b) \cdot_M c = a \cdot_M (b \cdot_M c)
    \end{equation*}
    \begin{equation*}
      1_M \cdot_M a = a = a \cdot_M 1_M
    \end{equation*}
    Neutralni element $1_M$ i binarnu operaciju $\cdot_M$ uglavnom ćemo pisati
    kao $1$ i $\cdot$ osim ako iz konteksta neće biti jasno na kojem monoidu
    su definirani.
    Kada monoid promatramo kao kategoriju tada su objekti skupovi, a
    za dva monoida $M, N$ definiramo strelicu $M \xrightarrow{\delta} N$ kao
    funkciju za koju vrijedi da za svaki $a, b \in M$ vrijedi:
    \begin{equation*}
      \delta(a \cdot b) = \delta(a) \cdot \delta(b)
    \end{equation*}
    \begin{equation*}
      \delta(1) = 1
    \end{equation*}
    i zovemo ju \textbf{morfizam}.
    Pokažimo da je kompozicija dva morfizma morfizam.
    Neka su $a, b \in Obj_M$ i $M \xrightarrow{f} N \xrightarrow{g} P$, tada vrijedi:
    \begin{equation*}
      (g \circ f)(a \cdot b) = g(f(a \cdot b)) = g(f(a) \cdot f(b)) = g(f(a))
      \cdot g(f(b)) = (g \circ f)(a) \cdot (g \circ f)(b)
    \end{equation*}
    Za monoid $M$ definiramo identitetu $id_M$ kao standardnu funkciju
    identiteta, tj. za svaki $a \in Obj_M$ vrijedi:
    \begin{equation*}
      id_M(a) = a
    \end{equation*}
    Pokažimo sada da tako definirane strelice na monoidu zadovoljavaju
    svojstvo identiteta i asocijativnosti (\ref{def:kat_assoc}) za kategorije.
    Neka je $a \in Obj_M$ i $M \xrightarrow{f} N$, tada vrijedi:
    \begin{equation*}
      (id_N \circ f)(a) = id_N(f(a)) = f(a) = f(id_M(a)) = (f \circ id_M)(a)
    \end{equation*}
    te je svojstvo identiteta zadovoljeno.\\
    Neka su $M, N, P, R$ monoidi, $M \xrightarrow{f} N \xrightarrow{g} P \xrightarrow{h} R$
    i $a \in Obj_M$, tada vrijedi:
    \begin{equation*}
      ((h \circ g) \circ f)(a) = h(g(f(a))) = (h \circ (g \circ f))(a)
    \end{equation*}
    Zbog asocijativnosti kompozicije funkcija te monoide možemo promatrati kao
    kategorije. Kategoriju monoida označavamo sa \category{Mon}.
  
  \end{example}

  %%%%%%%%%%%%%%%%%%%%
  %% Pos Kategorija %%
  %%%%%%%%%%%%%%%%%%%%
  \begin{example}\ \\
  
   \noindent Neka je $S$ skup i $\leq_S \subseteq S \times S$ relacija na $S$. Uređeni par $(S,
    \leq_S)$ zovemo \textbf{parcijalno uređeni skup} ako za svaki $a, b, c \in S$
    vrijedi:
    \begin{itemize}
      \item $a \leq_S a$ (refleksivnost)
      \item ako $a \leq_S b$ i $b \leq_S a$ tada $a = b$ (anti-simetričnost)
      \item ako $a \leq_S b$ i $b \leq_S c$ tada $a \leq_S c$ (tranzitivnost)
    \end{itemize}
    Radi kraćeg zapisa pisat ćemo samo $\leq$ umjesto $\leq_S$ i
  govoriti o parcijalno uređenom skupu $S$ gdje je implicitno definirana
  binarna relacija $\leq$.
  Neka su $S, R$ dva parcijalno uređena skupa i neka je $R \xrightarrow{f} S$
  preslikavanje za koje vrijedi da za svaki $a, b \in S$ vrijedi:
  \begin{equation*}
    a \leq b \implies f(a) \leq f(b)
  \end{equation*}
  Tada kažemo da je $f$ \textbf{monotono preslikavanje}.
  Pokažimo da su monotona preslikavanja zatvorena na kompoziciju.
  Neka su $S \xrightarrow{f} R \xrightarrow{g} Q$ monotona preslikavanja i $a,
  b \in S$ takva da $a \leq b$. Tada vrijedi:
  \begin{equation*}
    a \leq b \implies f(a) \leq f(b) \implies g(f(a)) \leq g(f(b))
  \end{equation*}
  tj.
  \begin{equation*}
    a \leq b \implies (g \circ f)(a) \leq (g \circ f)(b)
  \end{equation*}
  Identiteta na parcijalno uređenom skupu $S$ definirana je kao standardna
  funkcijska identiteta. Kao i u prethodnom primjeru pokaže se da vrijedi:
  \begin{equation*}
    (id_R) \circ f = f = f \circ id_S
  \end{equation*}
  \begin{equation*}
    (h \circ g) \circ f = h \circ (g \circ f)
  \end{equation*}
  pa možemo govoriti o kategoriji \category{Pos(S)} gdje su objekti parcijalno
  uređeni skupovi a strelice monotona preslikavanja.\\
  \end{example}

  \begin{definition}
    \textbf{Gornja međa} podskupa $T$ parcijalno uređenog skupa $(S, \leq)$ je $v \in S$ t.d. $\forall t \in T, t \leq v$
    \end{definition}
  \begin{definition}
    \textbf{Supremum} ili \textbf{najmanja gornja međa} podskupa $T$ parcijalno uređenog skupa $(S, \leq)$ je $v \in S$ t.d. je $v$ gornja međa skupa $T$ i vrijedi: ako $w \in S$ ima svojstvo da je $t \leq w$ za svaki $t \in T$, tada je $v \leq w$.
    \end{definition}
  
  %%%%%%%%%%%%%
  %% Haskell %%
  %%%%%%%%%%%%%
  Haskell je funkcijski programski jezik nazvan po logičaru Haskellu Curryju. U
  Haskellu \textbf{tip} intuitivno možemo shvatiti kao skup, npr.
  \codei{Char} je skup svih Unicode znakova (\codei{'A'},
  \codei{')'}, \codei{'^'}), \codei{Bool} sadrži samo dva elementa
  \codei{\{True, False\}}, \codei{Integer} je beskonačni skup koji sadrži
  sve cijele brojeve, \codei{Void} je prazan skup ($\emptyset$).
  Kada u Haskellu zapišemo da je $x$ Integer, tj:
  \code{
    x :: Integer
  }
  to znači da je $x$ element skupa cijelih brojeva. Poseban tip je
  \codei{()} koji nazivamo unit i čija je jedina vrijednost
  \codei{()}.

  Haskellove funkcije ne možemo poistovjetiti s matematičkim funkcijama jer
  Haskellove funkcije trebaju izvršiti neki kod - što nije problem ukoliko se
  može doći do rezultata u konačno mnogo koraka, ali ponekad to nije slučaj. Zbog "Halting
  problema"\footnote{Ako je zadan program i ulaz za taj program, odredite hoće
    li računanje tog program stati.
    \url{http://mathworld.wolfram.com/HaltingProblem.html}} ne možemo se
  ograničiti samo na funkcije kod kojih se može doći do rezultata u konačno
  mnogo koraka pa se u svaki Haskellov tip dodaje posebna vrijednost: ⊥ (dno).
  Ona označava da računanje neće stati, pa tako funkcija $f:
  Bool \to Bool$  u Haskellu definirana sa:
  \code{
    f :: Bool -> Bool
  }
  može vratiti \codei{True}, \codei{False} ili ⊥. Funkcije koje mogu vratiti ⊥
  zovemo parcijalne funkcije.
  Zbog specijalnog znaka ⊥ kategoriju Haskellovih funkcija i tipova razlikujemo
  od \category{Set} i označavamo sa \category{Hask}. Tom distinkcijom dolazimo
  do nekih komplikacija \footnote{seq - \url{https://wiki.haskell.org/Seq}}
  ali su Nils Anders Danielsson, John Hughes, Patrik Jansson i Jeremy Gibbons
  u \textit{Fast and Loose Reasoning is Morally Correct} (\ref{bib:fast-loose}) pokazali
  da za potrebe ovog rada možemo ignorirati znak ⊥ i njegove posljedice. Uzevši to u obzir, dajemo idealiziranu definiciju \category{Hask}-a.\\
  %%%%%%%%%%%%%%%%%%%%%
  %% Hask Kategorija %%
  %%%%%%%%%%%%%%%%%%%%%
  
	\begin{example}\ \\
	
  \noindent \category{Hask} je kategorija Haskellovih tipova i funkcija. U
  \category{Hask} kategoriji objekti su Haskellovi tipovi koje označavamo velikim
  slovima:
  \code{A, B, C, ...}.
  Strelice u \category{Hask}-u su Haskellove funkcije koje označavamo malim
  slovima:
  \code{f, g, h, ...}.
  Strelicu $A \xrightarrow{f} B$ u \category{Hask}-u zapisujemo:
  \code{f :: A -> B}.
  Funkcije
  \code{
      f :: A -> B, g :: A -> B
      }
  su jednake ako za svaki \texttt{x} vrijedi:
    \code{ f x = g x }.
  Kompoziciju ($A \xrightarrow{f \circ g} C$) zapisujemo:
    \code{(f.g) :: A -> C}
    i definiramo kao standardnu funkcijsku kompoziciju, tj.:
    \code{(f.g) x = f (g x)}
  pa se lako pokaže da vrijedi svojstvo asocijativnosti.
  Identiteta u \category{Hask}-u dana je funkcijom:
    \code{ id x = x }
  te se lako vidi da vrijedi:
    \code{ id.f = f = id.f }.\\
  \end{example}

  %%%%%%%%%%%%%%%%%%%%%%%%%%%%%%%%%%%%%%%%%%%%%%%%%%%%%%%
  %% Kategorija gdje strelica nije standardna funkcija %%
  %%%%%%%%%%%%%%%%%%%%%%%%%%%%%%%%%%%%%%%%%%%%%%%%%%%%%%%
	\begin{example}\ \\
	
    \noindent Pokažimo sada primjer neke kategorije $G$ gdje su objekti konačni skupovi,
    a strelica između objekata ne mora biti uobičajena funkcija.\\
    Neka su $A, B$ konačni skupovi, strelica $A \xrightarrow{f} B$ je
    proizvoljna funkcija: $f:A \times B \rightarrow \mathbb{R}$
    Za strelice $A \xrightarrow{f} B \xrightarrow{g} C$ definiramo: $g \circ f:A \times C \rightarrow \mathbb{R}$
    sa
    \begin{equation*}
      (g \circ f)(a, c) = \sum_{y \in B}f(a, y)g(y,c)
    \end{equation*}
    Lako se vidi da je to kategorija.\\
\end{example}

  %%%%%%%%%%%%%%%%%%%%%%%
  %% Dualna kategorija %%
  %%%%%%%%%%%%%%%%%%%%%%%
	\begin{example}\ \\
	
    \noindent Neka je $G$ neka kategorija. Tada definiramo kategoriju $G^{op}$ koja ima
    iste objekte kao i $G$, a za svaku strelicu $A \xrightarrow{f} B$ u $G$ 
    dana je dualna strelica $B \xrightarrow{f^{op}} A$ u $G^{op}$.
    Za strelice $A \xrightarrow{f} B \xrightarrow{g} C$ kompozicija strelica
    u $G^{op}$ definirana je sa:
    \begin{equation*}
      f^{op} \circ^{op} g^{op} = (g \circ f)^{op}
    \end{equation*}
    Pokažimo da tako definirana struktura $G^{op}$ zadovoljava svojstvo
    asocijativnosti i identiteta, tj. da je tako zadana struktura stvarno
    kategorija.
    Neka su $A \xrightarrow{f} B \xrightarrow{g} C \xrightarrow{h} D$, tada je:
    \begin{equation*}
      id^{op}_B \circ^{op} f^{op} = (id_B \circ f)^{op} = (f \circ id_A)^{op} =
      f^{op}
    \end{equation*}
    i
    \begin{align*}
      (h^{op} \circ^{op} g^{op}) \circ^{op} f^{op} &= (h \circ g)^{op} \circ^{op} f^{op}\\
      &= ((h \circ g) \circ f)^{op}\\
      &= (h \circ (g \circ f))^{op}\\
      A
      &= h^{op} \circ^{op} (g \circ f)^{op}\\
      &= h^{op} \circ^{op} (g^{op} \circ^{op} f^{op})
    \end{align*}
    Pa je $G^{op}$ kategorija.\\
\end{example}

\noindent U kasnijim razmatranjima bit će nam potreban pojam podkategorije.\\

  \begin{definition}\ \\
  
   \noindent Za kategoriju \category{G} kažemo da je podkategorija kategorije
    \category{C} ako vrijedi
    \begin{itemize}
      \item $Obj_\category{G} \subseteq Obj_\category{C}$
      \item za sve $A, B \in \category{G}, \category{G}[A, B] \subseteq
        \category{C}[A, B]$
      \item kompozicija i identitata su jednaki u \category{G} i \category{C}
    \end{itemize}
    Kažemo da je podkategorija \textbf{puna} ako za svaki $A, B \in \category{G}$ vrijedi
    $\category{G}[A, B] = \category{C}[A, B]$\\
    
  \end{definition}
  
\noindent Primijetimo da je puna podkategorija jedinstveno zadana svojim objektima.
  %%%%%%%%%%%%%%%%%%%%%%%%
  %%%%%%%%%%%%%%%%%%%%%%%%
  %%%%%%% DIJAGRAM %%%%%%%
  %%%%%%%%%%%%%%%%%%%%%%%%
  %%%%%%%%%%%%%%%%%%%%%%%%
  \newpage
  \section{Dijagram}
  U teoriji kategorija dijagrame koristimo kao reprezentaciju jednadžbi.
  Za kategoriju \category{G} i $A, B \in Obj_G$ već smo vidjeli da
  preslikavanje $f \in \category{G}[A, B]$ označavamo dijagramom:
  \begin{align*}
    A \xrightarrow{f} B
  \end{align*}
  Pogledajmo sljedeći dijagram s tri strelice:
  \begin{center}
    \begin{tikzpicture}[every node/.style={midway}]
      \matrix[column sep={5em,between origins}, row sep={2em}] at (0,0)
      {
        \node(A) {$\bullet$};\\
        \node(B) {$\bullet$}; & \node(C) {$\bullet$}; \\
      };

      \draw[->] (A) -- (B) node[anchor=east] {$f$};
      \draw[->] (B) -- (C) node[anchor=north]  {$g$};
      \draw[->] (A) -- (C) node[anchor=south] {$h$};
    \end{tikzpicture}
  \end{center}
  Primijetimo da objektima nismo dali imena (zato što nam to za ovaj primjer nije
  bitno), ali to ne znači da su ta tri objekta jednaka. Ako vrijedi:
  \begin{align*}
    g \circ f = h
  \end{align*}
  tada kažemo da dijagram \textbf{komutira}.
  Zakon asocijativnosti za $f, g, h$ dijagramom možemo prikazati kao:
  \begin{center}
    \begin{tikzpicture}[every node/.style={midway}]
      \matrix[column sep={5em,between origins}, row sep={2em}] at (0,0)
      {
        \node(A) {$A$}; & \node(B) {$B$}; \\
        \node(C) {$C$}; & \node(D) {$D$}; \\
      };
      \draw[->] (A) -- (B) node[anchor=south] {$f$};
      \draw[->] (B) -- (C) node[anchor=north] {$g$};
      \draw[->] (C) -- (D) node[anchor=north] {$h$};
    \end{tikzpicture}
  \end{center}


  Kao i u mnogim granama matematike, u teoriji kategorija ponekad želimo
  pokazati da su dvije stvari jednake. Rijetko želimo pokazati da su dva
  objekta jednaka; češće ćemo pokazivati jednakost strelica i za to ćemo
  koristiti tehniku zvanu praćenje dijagrama (eng. diagram chasing).
  Dijagram:

  \begin{center}
    \begin{tikzpicture}[every node/.style={midway}]
      \matrix[column sep={5em,between origins}, row sep={2em}] at (0,0)
      {
        & \node(B) {$\bullet$}; &\\
        \node(A) {$\bullet$}; && \node(C) {$\bullet$}; \\
        & \node(D) {$\bullet$}; & \\
      };
      \draw[->] (A) -- (B) node[anchor=south] {$f$};
      \draw[->] (B) -- (C) node[anchor=south] {$g$};
      \draw[->] (B) -- (D) node[anchor=east] {$h$};
      \draw[->] (A) -- (D) node[anchor=north] {$k$};
      \draw[->] (D) -- (C) node[anchor=north] {$l$};
    \end{tikzpicture}
  \end{center}
  ima četiri neimenovana objekta, pet strelica ($f, g, h, k, l$) i pet
  strelica nastalih kompozicijom:
  \begin{align*}
    g \circ f, h \circ f, l \circ h \circ f, l \circ h, l \circ k
  \end{align*}
  Primijetimo da neke od tih strelica mogu biti jednake.
  Ovaj dijagram ima tri ćelije: vanjsku $(f, k, l, g)$, lijevi
  unutarnji trokut $(f, h, k)$ i desni unutarnji trokut $(h, g, l)$.
  Neke od tih ćelija mogu komutirati:
  \begin{itemize}
      \item lijevi trokut komutira ako vrijedi $h \circ f = k$
      \item desni trokut komutira ako vrijedi $l \circ h = g$
      \item vanjska ćelija komutira ako vrijedi $g \circ f = l \circ k$
  \end{itemize}
  Praćenje dijagrama je proces u kojem pokazujemo da neka ćelija komutira pomoću
  činjenica da neke druge ćelije komutiraju te nekih drugih svojstava dijagrama.\\

\begin{example}\ \\

    \noindent Pokažimo za prethodni dijagram da ukoliko unutarnji trokuti
    komutiraju, tada vanjska ćelija komutira tj. ako vrijedi: $h \circ f = k$ i $ l \circ h = g$.
    Tada vrijedi: $g \circ f = l \circ k$.\\
    Pokažimo to algebarski:$g \circ f = (l \circ h) \circ f \overset{(\ref{def:kat_assoc})}{=} l \circ (h \circ f) = l \circ k$.
    Tehnikom praćenja dijagrama i uočavanjem da su neke kompozicije jednake
    imamo:
    \begin{center}
      \begin{tikzpicture}[every node/.style={midway}]
        \matrix[column sep={3em,between origins}, row sep={1em}, ampersand replacement=\&] at (0,0)
        {
          \& \node(B) {$\bullet$}; \&\&\&\& \node(E) {$\bullet$}; \&\&\&\&\&\\
          \node(A) {$\bullet$}; \&\& \node(C) {$\bullet$}; \& \node(EQ1) {$=$};
          \& \node(D) {$\bullet$}; \&\& \node(G) {$\bullet$}; \& \node(EQ2) {$=$};
          \& \node(I) {$\bullet$}; \&\& \node(K) {$\bullet$};\\
          \&\&\&\&\& \node(F) {$\bullet$}; \&\&\&\& \node(J) {$\bullet$}; \& \\
        };
      \draw[->] (A) -- (B) node[anchor=south] {$f$};
      \draw[->] (B) -- (C) node[anchor=south] {$g$};
      \draw[->] (D) -- (E) node[anchor=south] {$f$};
      \draw[->] (E) -- (F) node[anchor=east] {$h$};
      \draw[->] (F) -- (G) node[anchor=north] {$l$};
      \draw[->] (I) -- (J) node[anchor=north] {$k$};
      \draw[->] (J) -- (K) node[anchor=north] {$l$};
      \end{tikzpicture}
    \end{center}
\end{example}

  Kod jednostavnijih primjera ne vidi se prednost korištenja tehnike dijagrama.
  Pokažimo sada na malo kompliciranijem primjeru kako se tehnikom dijagrama
  intuitivnije može objasniti komutacija dijagrama.\\

\begin{example}\ \\

    \noindent Ako u dijagramu:
    \begin{center}
      \begin{tikzpicture}[every node/.style={midway}]
        \matrix[column sep={3em,between origins}, row sep={2em}, ampersand replacement=\&] at (0,0)
        {
          \& \node(B) {$\bullet$};\&\&  \node(C) {$\bullet$}; \&\\
          \node(A) {$\bullet$}; \&\& \node(E) {$\bullet$}; \&\& \node(D) {$\bullet$};\\
          \&\& \node(F) {$\bullet$}; \&\&\\
        };
      \draw[->] (A) -- (B) node[anchor=south] {$f$};
      \draw[->] (B) -- (C) node[anchor=south] {$g$};
      \draw[->] (C) -- (D) node[anchor=south] {$h$};
      \draw[->] (A) -- (E) node[anchor=south] {$i$};
      \draw[->] (E) -- (D) node[anchor=south] {$j$};
      \draw[->] (B) -- (E) node[anchor=south] {$m$};
      \draw[->] (E) -- (C) node[anchor=south] {$n$};
      \draw[->] (A) -- (F) node[anchor=north] {$k$};
      \draw[->] (F) -- (D) node[anchor=north] {$l$};
      \end{tikzpicture}
    \end{center}

    \noindent komutiraju četiri unutarnja trokuta, pokažimo da komutira i vanjska
    ćelija. Cilj nam je pokazati da vrijedi: $h \circ g \circ f = l \circ k$.\\
    
    \noindent Algebarski: $h \circ g \circ f = h \circ (n \circ m) \circ f = (h \circ n) \circ (m \circ f) = j \circ i = l \circ k$.\\
      
    \noindent Pokažimo to tehnikom praćenja dijagrama.
    \begin{center}
      \begin{tikzpicture}[every node/.style={midway}]
        \matrix[column sep={3em,between origins}, row sep={2em}, ampersand replacement=\&] at (0,0)
        {
          \node(EQ3) {$\quad$}; \&\& \node(B) {$\bullet$};\&\&  \node(C) {$\bullet$}; \&\&
          \&\& \node(B1) {$\bullet$};\&\&  \node(C1) {$\bullet$}; \&\\
          \&\node(A) {$\bullet$}; \&\& ; \&\& \node(D) {$\bullet$}; \& \node(EQ1) {$=$};
          \& \node(A1) {$\bullet$}; \&\& \node(E1) {$\bullet$}; \&\& \node(D1)
          {$\bullet$}; \& \node(EQ2) {$=$};\\
        };
      \draw[->] (A) -- (B) node[anchor=south] {$f$};
      \draw[->] (B) -- (C) node[anchor=south] {$g$};
      \draw[->] (C) -- (D) node[anchor=south] {$h$};

      \draw[->] (A1) -- (B1) node[anchor=south] {$f$};
      \draw[->] (C1) -- (D1) node[anchor=south] {$h$};
      \draw[->] (B1) -- (E1) node[anchor=south] {$m$};
      \draw[->] (E1) -- (C1) node[anchor=south] {$n$};

      \end{tikzpicture}
    \end{center}
    \begin{center}
      \begin{tikzpicture}[every node/.style={midway}]
        \matrix[column sep={3em,between origins}, row sep={2em}, ampersand replacement=\&] at (0,0)
        {
          \node(EQ2) {$=$}; \& \node(A) {$\bullet$}; \&\& \node(E) {$\bullet$}; \&\& \node(D)
          {$\bullet$}; \& \node(EQ1) {$=$};
          \& \node(A1) {$\bullet$}; \&\& \&\& \node(D1) {$\bullet$}; \&\\
          \&\&\&\&\&\&\&\&\& \node(F1) {$\bullet$}; \&\&\\
        };
      \draw[->] (A) -- (E) node[anchor=south] {$i$};
      \draw[->] (E) -- (D) node[anchor=south] {$j$};
      \draw[->] (A1) -- (F1) node[anchor=south] {$k$};
      \draw[->] (F1) -- (D1) node[anchor=south] {$l$};
      \end{tikzpicture}
    \end{center}

\end{example}
  %%%%%%%%%%%%%%%%%%%
  %% Monoik i epik %%
  %%%%%%%%%%%%%%%%%%%
  \newpage
  \section{Monoik i epik}
  U ovom poglavlju proširujemo terminologiju i proučavamo posebne slučajeve
  strelica kako bismo došli do definicije izomorfizma.\\
  
  \begin{definition}\ \\
  
    \noindent U proizvoljnoj kategoriji \category{G} strelicu: $B \xrightarrow{m} A$
    za koju vrijedi da za svaki par strelica
     $ X \xbigtoto[g]{f} B$
    vrijedi: $m \circ f = m \circ g \implies f = g$
    zovemo \textbf{monoik}.\\
  \end{definition}
  
\begin{example}\ \\

    \noindent Neka je \category{G} neka kategorija na skupovima i neka su strelice
    totalne funkcije između skupova.
    Ako je $B \xrightarrow{m} A$ injektivna (kao funkcija) tada je $m$ monoik (kao
    strelica). Pokažimo to.\\    
    \noindent Pretpostavimo da vrijedi: $ m \circ f = m \circ g$ \label{me:pr:1}
    za neki paralelni par strelica $A \xbigtoto[g]{f} B$. Kako su $f$ i $g$ totalne funkcije dovoljno je pokazati da za svaki $x \in X$ vrijedi: $f(x) = g(x)$. Imamo:
     $m(f(x)) = (m \circ f)(x) \overset{(\ref{me:pr:1})}{=} (m \circ g)(x) = m(g(x))$.
    No, kako je $m$ injektivna za $a, b \in B$ imamo:
     $m(a) = m(b) \implies a = b$
   	pa je $m$ monoik.\\
\end{example}

  \begin{definition}\ \\
  
    \noindent U proizvoljnoj kategoriji \category{G} strelicu: $A \xrightarrow{e} B$ za koju vrijedi da za svaki par strelica $B \xbigtoto[g]{f} X$ vrijedi: $f \circ e = g \circ e \implies f = g$ zovemo \textbf{epik}.
  \end{definition}
  
  Pokažimo sada na sličnom primjeru kao i ranije kako se epik ponaša na
  kategoriji nad skupovima.

\begin{example}\ \\
	
	\noindent Neka je \category{G} neka kategorija na skupovima i neka su strelice
    totalne funkcije između skupova.
    Ako je $A \xrightarrow{e} B$ surjektivna (kao funkcija) tada je epik (kao
    strelica).
    Pretpostavimo da vrijedi:
    \begin{align*} \label{me:pr:2}
      f \circ e = g \circ e
    \end{align*}
    za neki paralelni par strelica
    \begin{align*}
      B \xbigtoto[g]{f} X
    \end{align*}
    Kako su $f$ i $g$ totalne funkcije dovoljno je pokazati da za $x \in X$
    vrijedi:
    \begin{align*}
      f(x) = g(x)
    \end{align*}
    Kako je $e$ surjektivna znamo da za $b \in B$ postoji $a \in A$ takav da
    vrijedi:
    \begin{align*}
      b = e(a)
    \end{align*}
    Imamo:
    \begin{align*}
      f(b) = f(e(a)) = (f \circ e)(a) = (g \circ e)(a) = g(e(a)) = g(b)
    \end{align*}
    Pa je $e$ epik.\\
\end{example}

    Prethodni primjeri pokazuju da za "dovoljno lijepe" kategorije vrijedi:
    \begin{center}
      injekcija $\implies$ monoik \qquad i \qquad surjekcija $\implies$ epik.
    \end{center}
    No, za neke kategorije "injektivna strelica" ili "surjektivna strelica"
    nemaju smisla tako da takva intuitivna interpretacija ima smisla samo za
    "dovoljno lijepe" kategorije. Pravilnije bi bilo gledati na monoik i epik
    kao na strelice koje se mogu "poništiti" na jednoj strani. Ako pak strelica
    ima jednostrani inverz tada dobivamo posebnu klasu monoika i epika.\\
    
    \begin{definition}\ \\
    
      \noindent Ako za strelice:
       $B \xrightarrow{s} A, A \xrightarrow{r} B$
      vrijedi:
       $r \circ s = id_B$
      tada $s$ nazivamo \textbf{sekcija}, a $r$ \textbf{retrakcija}.
    \end{definition}
    
    Lako se pokaže da je svaka sekcija monoik i svaka retrakcija epik.
    
    \begin{definition}\ \\
    
    \noindent \textbf{Bimorfizam} je strelica koja je monik i epik.\\
    \end{definition}
    
    \begin{definition}\ \\
    
    \noindent Za strelicu $A \xrightarrow{f} B$ kažemo da je \textbf{izomorfizam} ako
    postoji strelica $B \xrightarrow{g} A$ takva da vrijedi:
      \begin{align*}
        g \circ f = id_A \wedge f \circ g = id_B
      \end{align*}
    \end{definition}
    
    \begin{definition}\ \\
    
    \noindent Za dva objekta $A$ i $B$ kažemo da su \textbf{izomorfni} ako postoji izomorfizam
    $A \xrightarrow{} B$.\\
    \end{definition}

    Lako se pokaže da ukoliko je strelica sekcija i epik ili retrakcija i
    monoik tada je i izomorfizam.
    Grafički to ilustriramo na sljedećoj skici:
    \begin{center}
    \begin{tikzpicture}[every node/.style={midway}]
      \matrix[column sep={5em,between origins}, row sep={2em}] at (0,0)
      {
        & \node(S) {Strelica}; &\\
        \node(M) {Monoik}; && \node(E) {Epik}; \\
        & \node(B) {Bimorfizam}; & \\
        \node(SM) {Sekcija}; && \node(SE) {Retrakcija};\\
        & \node(I) {Izomorfizam}; & \\
      };
      \draw[->] (S) -- (M) node[anchor=north] {};
      \draw[->] (S) -- (E) node[anchor=north] {};
      \draw[->] (E) -- (SE) node[anchor=north] {};
      \draw[->] (E) -- (B) node[anchor=north] {};
      \draw[->] (M) -- (SM) node[anchor=north] {};
      \draw[->] (M) -- (B) node[anchor=north] {};
      \draw[->] (SM) -- (I) node[anchor=north] {};
      \draw[->] (SE) -- (I) node[anchor=north] {};
      \draw[->] (B) -- (I) node[anchor=north] {};
    \end{tikzpicture}
    \end{center}
    
  %%%%%%%%%%%%%%%%%%%%%%%%%%%%%%%
  %% Kategorijski konstruktori %%
  %%%%%%%%%%%%%%%%%%%%%%%%%%%%%%%
  \newpage
  \section{Kategorijski konstrukori}
  U ovom poglavlju istražujemo neke osnovne kategorijske
  konstruktore, tj. objekte koji zadovoljavaju neka pravila definirana u teoriji
  kategorija. Kako u jeziku kategorija ne gledamo unutrašnju strukturu objekata
  svi koncepti moraju biti definirani pomoću relacija između objekata.
  \subsection{Inicijalni i finalni objekti}

  \begin{definition}\ \\
  
    \noindent Za objekt $S$ u kategoriji \category{G} kažemo da je \textbf{inicijalni} ako za svaki
    objekt $A$ postoji jedinstvena strelica $S \xrightarrow{} A$.\\
  \end{definition}
  
  \begin{definition}\ \\
  
    \noindent Za objekt $S$ u kategoriji \category{G} kažemo da je \textbf{finalni} ako za svaki
    objekt $A$ postoji jedinstvena strelica $A \xrightarrow{} S$.\\
  \end{definition}
  
  Lako se pokaže da ukoliko je objekt $I$ finalni objekt u kategoriji \category{G} tada je $I$ inicijalni objekt u $\category{G}^{op}$.\\
  
  \begin{example}\ \\
  
    \noindent Neka je \category{Set} kategorija skupova i neka je $1 = \{x\}$ skup s jednim elementom. Za svaki skup $A$ postoji jedinstvena strelica $A \xrightarrow{} 1$ (funkcija koja preslikava sve u $x$) pa je $1$ finalni objekt za \category{Set}. Za svaki skup $A$ postoji jedinstvena strelica $\emptyset \xrightarrow{} A$ (gdje je $\emptyset$ prazan skup) pa je $\emptyset$ inicijalni objekt za \category{Set}.\\
  
  \end{example}
  
  \begin{example}\ \\
  
  \noindent \codei{Void} je inicijalni objekt u \category{Hask}.
    Zbog polimorfizma \footnote{Parametrizirani polimorfizam omogućava da se
      funkcije pišu generalno bez ovisnosti o tipu, opširnije o tome u
      poglavlju "Funktori". Za sada možemo funkciju \codei{absurd :: -> a}
      interpretirati kao skup funkcija \{Void $\to$ a $|$ a je tip u
      \category{Hask}\} čije pozivanje ovisi o kodomeni.} možemo definirati polimorfnu funkciju:
    \code{
      absurd :: Void -> a
    }
    gdje je \codei{a} bilo koji tip u \category{Hask}.\\
  
  \end{example}
  
  \begin{example}\ \\
  
  \noindent \codei{()} je finalni objekt u \category{Hask}.
    Možemo definirati polimorfnu funkciju \codei{unit :: a -> ()} kao:
    \code{
      unit x = ()
    }
  \end{example}
  
  Kategorija \category{C} može imati i finalni i inicijalni objekt, a ako ima
  oboje onda oni ne moraju biti isti. Objekt koji je i finalni i inicijalni ponekad
  zovemo \textbf{nulti} objekt. Lako se pokaže da ukoliko kategorija ima dva inicijalna
  objekta tada su oni izomorfni. Stoga ima smisla govoriti o inicijalnom objektu
  kategorije te analogno o finalnom objektu.
  
  \subsection{Produkti i koprodukti}
  Produkt u teoriji kategorija je generalizacija Kartezijskog produkta na
  skupovima. Prisjetimo se, ako su $A, B$ skupovi tada je Kartezijev produkt ta
  dva skupa definiran sa: $A \times B = \{ (a, b) | a \in A, b \in B\}$.\\
  
  \noindent Primijetimo da je uz takvu definiciju Kartezijevog produkta prirodno
  definirati dvije projekcije:
    $p_A : A \times B \to A, p_A(a, b) = a$; 
    $p_B : A \times B \to B, p_B(a, b) = b$.\\

  \noindent Ako su \codei{a} i \codei{b} proizvoljni tipovi u \category{Hask} te dvije
  funkcije zovemo \codei{fst} i \codei{snd}.
  \begin{mcode}
    fst :: (a, b) -> a
    fst (x, y) = x

    snd :: (a, b) -> b
    snd (x, y) = y
  \end{mcode}
  Pa na razini kategorija definiramo prvi konstruktor.\\
  
  \begin{definition}\ \\
  
    \noindent Za $A, B \in Obj_C$ \textbf{grananje prema paru $A, B$} je objekt $X \in
    Obj_C$
    zajedno sa strelicama:
  \begin{center}
    \begin{tikzpicture}[every node/.style={midway}]
      \matrix[column sep={3em,between origins}, row sep={2em}] at (0,0)
      {
        && \node(A) {$A$}; \\
        \node(X) {$X$};\\
        && \node(B) {$B$};\\
      };
      \draw[->] (X) -- (A) node[anchor=south] {};
      \draw[->] (X) -- (B) node[anchor=south] {};
    \end{tikzpicture}
  \end{center}
  \end{definition}
  
  \begin{definition}\ \\
  
    \noindent Za $A, B \in Obj_C$ \textbf{grananje od para $A, B$} je objekt $X \in
    Obj_C$
    zajedno sa strelicama:
  \begin{center}
    \begin{tikzpicture}[every node/.style={midway}]
      \matrix[column sep={3em,between origins}, row sep={2em}] at (0,0)
      {
        \node(A) {$A$}; \\
        && \node(X) {$X$};\\
        \node(B) {$B$};\\
      };
      \draw[->] (A) -- (X) node[anchor=south] {};
      \draw[->] (B) -- (X) node[anchor=south] {};
    \end{tikzpicture}
  \end{center}
  \end{definition}
  
  Kada će iz konteksta biti jasno o kojem grananju se radi, govorit ćemo
  samo o grananju. Sada možemo definirati produkt.\\
  
  \begin{definition}\ \\
  
    \noindent Neka su $A, B \in Obj_C$. \textbf{Produkt} od $A$ i $B$ je grananje
  \begin{center}
    \begin{tikzpicture}[every node/.style={midway}]
      \matrix[column sep={3em,between origins}, row sep={2em}] at (0,0)
      {
        && \node(A) {$A$}; \\
        \node(S) {$S$};\\
        && \node(B) {$B$};\\
      };
      \draw[->] (S) -- (A) node[anchor=south] {$p_A$};
      \draw[->] (S) -- (B) node[anchor=north] {$p_B$};
    \end{tikzpicture}
  \end{center}
  sa svojstvom da za svako grananje:
  \begin{center}
    \begin{tikzpicture}[every node/.style={midway}]
      \matrix[column sep={3em,between origins}, row sep={2em}] at (0,0)
      {
        && \node(A) {$A$}; \\
        \node(X) {$X$};\\
        && \node(B) {$B$};\\
      };
      \draw[->] (X) -- (A) node[anchor=south] {$f_A$};
      \draw[->] (X) -- (B) node[anchor=north] {$f_B$};
    \end{tikzpicture}
  \end{center}
  postoji jedinstvena strelica $X \xrightarrow{m} S$ takva da dijagram:
  \begin{center}
    \begin{tikzpicture}[every node/.style={midway}]
      \matrix[column sep={3em,between origins}, row sep={2em}] at (0,0)
      {
        && \node(A) {$A$}; \\
        \node(X) {$X$}; && \node(S) {$S$};\\
        && \node(B) {$B$};\\
      };
      \draw[->] (X) -- (A) node[anchor=south] {$f_A$};
      \draw[->] (X) -- (B) node[anchor=north] {$f_B$};
      \draw[->] (X) -- (S) node[anchor=south] {$m$};
      \draw[->] (S) -- (A) node[anchor=west] {$p_A$};
      \draw[->] (S) -- (B) node[anchor=west] {$p_B$};
    \end{tikzpicture}
  \end{center}
  komutira. Tada $m$ zovemo \textbf{mediator} za grananje na $X$.\\
  \end{definition}
  
  
  \noindent Primijetimo dvije stvari:
  \begin{itemize}
    \item produkt nije samo objekt, produkt je objekt i dvije strelice
    \item mediator je jedinstveni za grananje na $X$
  \end{itemize}
  
  Općenito, definicija produkta se može generalizirati na više od dva člana, ali ona ne
  doprinosi razmatranju o ovome radu, pa izjednačavamo definiciju $\mathbb{binarnog produkta}$ i
  $\mathbb{produkta}$. Za kategoriju C kažemo da ima produkt (binarni produkt) ukoliko
  za svaki $A, B \in Obj_C$ postoji produkt (binarni produkt).
  U \category{Hask}-u možemo definirati funkciju (višeg reda)
  \codei{factorizer} koja će za bilo koji tip c i dvije projekcije $c
  \xrightarrow{p} a$ i $c \xrightarrow{q} b$ vratiti jedinstveni mediator.
  \begin{mcode}
    factorizer :: (c -> a) -> (c -> b) -> (c -> (a, b))
    factorizer p q = \x -> (p x, q x)
  \end{mcode}
  Analogno definiramo i koprodukt.\\

  \begin{definition}\ \\
  
   \noindent Neka su $A, B \in Obj_C$. \textbf{Koprodukt} od $A$ i $B$ je grananje
  \begin{center}
    \begin{tikzpicture}[every node/.style={midway}]
      \matrix[column sep={3em,between origins}, row sep={2em}] at (0,0)
      {
        \node(A) {$A$}; \\
        && \node(X) {$S$};\\
        \node(B) {$B$};\\
      };
      \draw[->] (A) -- (X) node[anchor=south] {$i_A$};
      \draw[->] (B) -- (X) node[anchor=north] {$i_B$};
    \end{tikzpicture}
  \end{center}
  sa svojstvom da za svako grananje:
  \begin{center}
    \begin{tikzpicture}[every node/.style={midway}]
      \matrix[column sep={3em,between origins}, row sep={2em}] at (0,0)
      {
        \node(A) {$A$}; \\
        && \node(X) {$X$};\\
        \node(B) {$B$};\\
      };
      \draw[->] (A) -- (X) node[anchor=south] {$f_A$};
      \draw[->] (B) -- (X) node[anchor=north] {$f_B$};
    \end{tikzpicture}

  \end{center}
  postoji jedinstvena strelica $S \xrightarrow{m} X$ takva da dijagram:
  \begin{center}
    \begin{tikzpicture}[every node/.style={midway}]
      \matrix[column sep={3em,between origins}, row sep={2em}] at (0,0)
      {
        \node(A) {$A$}; \\
        \node(S) {$S$}; && \node(X) {$X$};\\
        \node(B) {$B$};\\
      };
      \draw[->] (A) -- (X) node[anchor=south] {$f_A$};
      \draw[->] (B) -- (X) node[anchor=north] {$f_B$};
      \draw[->] (A) -- (S) node[anchor=east] {$i_A$};
      \draw[->] (B) -- (S) node[anchor=east] {$i_B$};
      \draw[->] (S) -- (X) node[anchor=south] {$m$};
    \end{tikzpicture}
  \end{center}
  komutira.\\
  \end{definition}

  \begin{definition}\ \\
  
    \noindent Za kategoriju \category{C} kažemo da je \textbf{Kartezijeva} (skraćeno:
    "\category{C} je \textbf{CC}") ako:
    \begin{itemize}
      \item sadrži inicijalni objekt,
      \item za svaki par $A, B \in Obj_C$ postoji Kartezijev produkt.\\
    \end{itemize}
  \end{definition}
  
  \begin{example}\ \\
  
    \noindent Za kategoriju \category{Set} već smo pokazali da sadrži inicijalni objekt
    $\emptyset$.
    Pokažimo sada da je Kartezijev produkt i produkt u kategorijskom smislu.
    Neka su $A$ i $B$ skupovi.
    Za proizvoljno grananje
    \begin{center}
      \begin{tikzpicture}[every node/.style={midway}, ampersand replacement=\&]
        \matrix[column sep={3em,between origins}, row sep={2em}] at (0,0)
        {
          \&\& \node(A) {$A$}; \\
          \node(X) {$X$};\\
          \&\& \node(B) {$B$};\\
        };
        \draw[->] (X) -- (A) node[anchor=south] {$f_A$};
        \draw[->] (X) -- (B) node[anchor=north] {$f_B$};
      \end{tikzpicture}
    \end{center}
    na X definiramo $X \xrightarrow{m} A \times B$ kao:
      $m(x) = (f_A(x), f_B(x))$
    za $x \in X$. Lako se pokaže da sljedeći dijagram komutira:
  \begin{center}
    \begin{tikzpicture}[every node/.style={midway}]
      \matrix[column sep={3em,between origins}, row sep={2em}, ampersand replacement=\&] at (0,0)
      {
        \&\& \node(A) {$A$}; \\
        \node(X) {$X$}; \&\& \node(S) {$A \times B$};\\
        \&\& \node(B) {$B$};\\
      };
      \draw[->] (X) -- (A) node[anchor=south] {$f_A$};
      \draw[->] (X) -- (B) node[anchor=north] {$f_B$};
      \draw[->] (X) -- (S) node[anchor=south] {$m$};
      \draw[->] (S) -- (A) node[anchor=west] {$p_A$};
      \draw[->] (S) -- (B) node[anchor=west] {$p_B$};
    \end{tikzpicture}
  \end{center}
  Ostaje nam pokazati jedinstvenost, tj. da je $m$ jedina takva funkcija.
  Pretpostavimo da postoji neka druga funkcija
    $X \xrightarrow{h} A \times B$
  takva da  vrijedi:
    $p_A \circ h = f_A$ i $p_B \circ h = f_B$.
  Za $x \in X$ imamo:
  \begin{align*}
    h(x) = (p_A(h(x), p_B(h(x))) = ((p_A \circ h )(x), (p_B \circ h)(x)) =
    (f_A(x), f_B(x)) = m(x).
  \end{align*}
  Stoga je \category{Set} jedna Kartezijeva kategorija.
	\end{example}
	
  \begin{example}
    \label{pos:infimum}
    Neka su $x, y$ objekti u $\category{Pos}$. Prema definiciji produkta, produkt od $x$ i $y$ je objekt $z$, zajedno sa strelicama
    $z \to y$ i $z \to x$. Ako uzmemo u obzir kontekst ($\category{Pos}$), $z \to y$ i $z \to x$ možemo zapisati i kao $z \leq y$ i $z \leq x$.
    Definicija produkta zahtjeva i da za bilo koji objekt $w$ iz $\category{Pos}$, $w \to y$ i $w \to x$ postoji strelica $w \to z$, tj.
    ako $w \leq y$ i $w \leq x$ vrijedi $w \leq z$. S obzirom da je $z \leq y$ i $z \leq x$, $z$ je infimum od $x$ i $y$.
    Ovime primjerom smo pokazali da se egistencija produkta u $\category{Pos}$ može svesti na egzistenciju infimuma.
    Primjetimo da iz prethodnog primjera možemo zaključiti da ukoliko kategorija $\category{Pos}$ nema infimum, da tada nema niti produkt. 
  \end{example}
  \newpage

  %%%%%%%%%%%%%%
  %% Funktori %%
  %%%%%%%%%%%%%%
  \section{Funktori}
  Do sada smo definirali kategorije, posebno i \category{Hask} kategoriju,
  proučavali različite vrste strelica, objekata i pokazali njihove ekvivalente
  u \category{Hask}-u. U ovom poglavlju bavimo se "preslikavanjima" između
  kategorija i demonstriramo praktičnu primjenu teorije kategorije u Haskellu.
  
  \begin{definition}\ \\
  
    \noindent Neka su \category{C} i \category{G} kategorije. \textbf{Funktor} $F$ iz
    \category{C} u \category{G} je preslikavanje koje:
    \begin{itemize}
      \item svakome $x \in Obj_\category{C}$ pridružuje $F(x) \in
        Obj_\category{G}$
      \item svakome $X \xrightarrow{f} Y \in Arw_\category{C}$ pridružuje
        $F(X) \xrightarrow{F(f)} F(Y) \in Arw_\category{G}$ tako da vrijedi:
        \begin{itemize}
          \item $F(id_X) = id_{F(X)}$ za $X \in Obj_\category{C}$
          \item $F(g \circ f) = F(g) \circ F(f)$ za sve $X \xrightarrow{f} Y$ i
            $Y \xrightarrow{g} Z$
        \end{itemize}
    \end{itemize}
  \end{definition}
  
  Funktori čuvaju strukturu kategorije, odnosno sve što je povezano u početnoj
  kategoriji ostat će povezano i u krajnjoj.\\
  
  \begin{definition}\ \\
  
    \noindent Neka je \category{C} proizvoljna kategorija i neka je $F$ funktor sa
    \category{C} u \category{C}. Tada kažemo da je $F$ \textbf{endofunktor} nad
    \category{C}.\\
  \end{definition}
  
  \begin{definition}\ \\
  
    \noindent Neka je \category{C} proizvoljna kategorija te neka je $F$ preslikavanje
    iz \category{C}  u \category{C} takvo da vrijedi:
    \begin{itemize}
      \item $x = F(x)$ za svaki $x \in Obj_\category{C}$
      \item $f = F(f)$ za svaki $f \in Obj_\category{C}$
    \end{itemize}
    Tada je $F$ endofunktor nad \category{C} koji zovemo \textbf{funktor identitete}.\\
  \end{definition}
  
  Funktori u Haskellu usko su povezani s gornjom definicijom funktora. 
  Da bismo mogli objasniti funktore u Haskellu na primjeru napravit ćemo malu digresiju i definirati
  jedan od najjednostavnijih parametriziranih tipova: \codei{Maybe}.
  
  \subsection{Maybe}
  Parametrizirani tip \codei{Maybe} služi za enkapsuliranje opcionalnih
  vrijednosti, na primjer pretpostavimo da želimo napisati funkciju u Haskellu koja će
  dohvatiti prvi element liste cijelih brojeva. Što će se dogoditi ako je lista prazna?

  Tip \codei{Maybe} omogućava nam da modeliramo tip koji može ali i ne mora
  imati vrijednost. Isto kao što tip \codei{Bool} može imati \codei{True}
  ili \codei{False} tako \codei{Maybe} može imati vrijednosti \codei{Just a} ili
  \codei{Nothing}. Primijetimo da, iako formalno \codei{Maybe} može imati dvije
  vrijednosti, prirodno je vrijednost \codei{Nothing} interpretirati kao da nismo dobili ništa (što može
  služiti za rješavanje takozvanih rubnih uvjeta).

  \codei{Maybe} je parametrizirani konstruktor tipa, što znači da pomoću njega možemo generirati nove tipove. Definicija \codei{Maybe} u
  Haskellu glasi:
  \begin{mcode}
    data Maybe a = Just a | Nothing
  \end{mcode}
  Vidimo da pomoću \codei{Maybe a}, gdje je \codei{a} bilo koji tip možemo
  konstruirati druge tipove koji mogu imati vrijednost ili \codei{Just a} ili
  \codei{Nothing}. Npr. \codei{Maybe Integer} može imati vrijednost \codei{Just
    Integer} ili \codei{Nothing}. Vratimo li se na primjer liste, to znači da naša funkcija
  koja bi vraćala prvi element liste cijelih brojeva ima tip:
  \begin{mcode}
    head' :: [Integer] -> Maybe Integer
  \end{mcode}
  Kada pozovemo \codei{head} na praznoj listi dobijemo \codei{Nothing}, dok na
  nepraznoj dobijemo \codei{Just Int}, npr.
  \begin{mcode}
    >> head' [2, 4, 6, 8, 10]
    Just 2
    >> head' []
    Nothing
  \end{mcode}
  
  \subsection{Funktori u Haskellu}
  Kako se funktor sastoji od dva dijela: preslikava objekte iz jedne kategorije u drugu
  i preslikava strelice iz prve kategorije u drugu, funktori u Haskellu su iz
  \category{Hask} u \category{Func}, gdje je \category{Func} podkategorija od
  \category{Hask}-a. Funktor liste ide iz \category{Hask} u \category{Lst},
  gdje \category{Lst} sadrži samo liste, tj.\codei{[T]}\footnote{
    \codei{[a]} je konstruktor tipa sa jednim parametrom,
    kao i \codei{Maybe a}}
  za bilo koji tip \codei{T}. Definicija funktora u Haskellu glasi:
  \begin{mcode}
    class Functor (f :: * -> *) where
      fmap :: (a -> b) -> f a -> f b
  \end{mcode}
  koji mora zadovoljavati:
  \begin{mcode}
    fmap id = id
    fmap (f . g) = fmap f . fmap g
  \end{mcode}
  Primijetimo da su ti uvjeti u skladu sa definicijom funktora u kategoriji.\\
  
 \begin{example}\ \\
 
    \noindent Pokažimo da je \codei{Maybe a} funktor.
    Primijetimo da, pošto je \codei{Maybe} konstruktor tipa (s jednim
    parametrom), bilo koji tip \codei{X} možemo poslati iz \category{Hask}-a i
    dobiti novi tip \codei{Maybe X}. Zaključujemo da \codei{Maybe}
    preslikava objekte iz jedne kategorije u drugu\footnote{Objekti u
      kategoriji \category{Hask} su tipovi}.

    Definirajmo sada \codei{fmap} kako bi bili zadovoljeni ostali uvjeti da
    \codei{Maybe a} bude funktor.
    \begin{mcode}
      instance Functor Maybe where
        fmap f (Just x) = Just (f x)
        fmap f Nothing = Nothing
    \end{mcode}
    Iz gornje definicije vidimo da za bilo koji \codei{A} $\xrightarrow{f}$
    \codei{B} $\in
    Arw_\category{Hask}$ vrijedi: \codei{Maybe A}
    $\xrightarrow{\texttt{fmap f}}$ \codei{Maybe B}.
    Lako se provjeri da sa tako definiranom funkcijom \codei{fmap} 
    \codei{Maybe a} jest funktor.
\end{example}

  Korisna intuicija kod Haskellovih funktora je da oni reprezentiraju tipove
  preko kojih se može mapirati: to mogu biti liste cijelih brojeva,
  balansirana stabla, opcionalne vrijednosti (\codei{Maybe a}) ili nešto drugo.

  Jednostavan primjer mapiranja preko liste cijelih brojeva može se pokazati
  korištenjem funkcije \codei{double} koja će udvostručiti cijeli broj.
  Definicija \codei{double} u Haskellu glasi:
  \begin{mcode}
    double :: Double -> Double
    double n = n * 2
  \end{mcode}
  Primijetimo da \codei{double} možemo gledati kao \codei{Double}
  $\xrightarrow{\texttt{double}}$ \codei{Double}. Kako je lista cijelih brojeva (\codei{[Double]}) funktor, možemo preslikati \codei{double} 
  u \codei{[Double]} $\xrightarrow{\texttt{fmap double}}$
  \codei{[Double]}. Npr.:
  \begin{mcode}
    >> fmap double [1.0..10.0]
    [2.0,4.0,6.0,8.0,10.0,12.0,14.0,16.0,18.0,20.0]
    \end{mcode}

\vspace{\baselineskip}

 	Time smo preko teorije kategorije funktora eliminirali potrebu za
  petljama koje su neizostavne kod imperativnih programskih jezika.
  \newpage


  \section{Kartezijanski zatvorene kategorije}
  U ovom poglavlju definiramo pojam kartezijanski zatvorene kategorija, vrstu kategorije koja ima ekspresivnu moć jednaku tipiziranom lambda računu.  Da bi definirali projam zatvorene kategorije nužno je definirati i pojam eksponenta, kategorijskog analogona funkcijskom prostoru između dva skupa. Nakon definicije kartezijanski zatvorenih kategorija navodimo nekoliko najbitinjih primjera. 
  \begin{definition}
    Neka je C kategorija koji ima produkt i neka su $Z, Y \in Obj_C$. \textbf{Eksponencijani objekt} je objekt $Z^Y$ koji ima svojstvo $Z^Y \times Y \xrightarrow{apply} Z$ ako za bilo koji objekt $X \in Obj_c$ i morfizam $X \times Y \xrightarrow{g} Z$ postoji jedinstveni morfizam $X \xrightarrow{\lambda g} Z^Y$ t.d.
    \begin{center}
      $X \times Y \xrightarrow{\lambda g \times Y} Z^Y \times Y \xrightarrow{apply} Z$ je $g$
    \end{center}
  \end{definition}
    Morfizam $apply$ se ponekad u literaturi zove i $eval$. U kategoriji \category{Set}, eksponencijalan objekt $Z^Y$ je skup svih funkcija $Y \to Z$ gdje jednostavno možemo vidjeti strukturu $apply$ funkcije:
    \begin{center}
      $Z^Y \times Y \xrightarrow{apply} Z$ pridružuje uređenom paru $(f, y)$ rezultat $f(y)$.
    \end{center}
    Slično, za preslikavanje $X \times Y \xrightarrow{g} Z$ možemo definirati morfizam $X \xrightarrow{\lambda g} Z^Y$ (tzv. currying) kao:
    \begin{center}
      $\lambda g(x)(y) = g(x, y)$
    \end{center}
  \begin{definition}
    Za kategoriju C kažemo da je \textbf{kartezijanski zatvorena kategorija}, skračeno \textbf{CCC} (eng. Cartesian closed category), ako vrijedi:
    \begin{itemize}
      \item kategorija \category{C} sadrži finalni objekt,
      \item za svaki par $A, B \in Obj_C$ postoji Kartezijev produkt $A \times B$ u C, sa projekcijama $\pi_1 : A \times B \to A$ i $\pi_2 : A \times B \to B$
      \item za svaki par $A, B \in Obj_C$ postoji eksponencijal $B^A$ u C.
    \end{itemize}
  \end{definition}
  Primijetimo da smo i prethodnu tvrdnju mogli iskazati pomoću prethodno definirane kartezijeve kategorije; kartezijeva kategorija C je kartezijanski zatvorena kategorija, ukoliko za svaki par $A, B \in Obj_C$ postoji eksponencijal $B^A$ u C.

  \primjer{Pokazali smo da je \category{Set} kategorija i da je $\emptyset$ finalni objekt za \cateogry{Set}. Postojanje eksponenta i njegove struktura je opisana  u 1.18. Pokažimo sad još da u \category{Set} postoji i kartezijev produkt.
  \begin{itemize}
    \item Za $A, B \in Obj_\category{Set}$, $A \times B = \{ (a, b) | a \in A, b \in B \}$, tj. $A \times B$ je standardni kartezijevski produkt nad skupovima. 
  \end{itemize}

  Iz čega zaključujemo da je \category{Set} CCC.
  }
  \begin{definition}
    Parcijalno uređeni skup B zovemo \textbf{Booleanova algebra} ako:
    \begin{itemize}
      \item za sve $x, y \in B$ postoji supremum $x \vee y$
      \item za sve $x, y \in B$ postoji infimum $x \wedge y$
      \item za sve $x, y, z \in b$ vrijedi: $x \vee (y \wedge z) = (x \wedge y) \vee (x \wedge z)$
      \item $B$ sadrži dva elementa, $0$ i $1$, gdje je $0$ najmanji element u B, a $1$ najveći
      \item svaki element $x \in B$ ima \textbf{komplement} $\neg x$, tj. $x \wedge \neg x = 0$ i $x \vee \neg x = 1$
    \end{itemize}
      Za Booleanovu algebru $B$ kažemo još i da je trivijalna ako $1 = 0$.
    \end{definition}
  Pokažimo sada da kategorija koja odgovara Booleanovoj algebri, kategorija $\category{Pos(B)}$ gdje su objekti Booleanov algebre, CCC. 
  \begin{example}
    Neka je $\category{Pos(B)}$ kategorija koja odgovara Booleanovoj algebri. Pokažimo da ona zadovoljava svojstva CCC-a. Finalni objekt je 1, a prema definiciji Booleanove algebre, za svaka dva elementa postoji infimum, a prema \ref{pos:infimum}, to je dovoljan uvjet za postojanje produkta.

    Preostalo nam je za svaki par $A, B \in Obj_{\category{Pos(B)}}$ pokazati postojanje eksponencijala $B^A$ u $\category{Pos(B)}$. Definiramo da je $B^A$ jednako $\neg A \vee B$. Da bi smo pokazali postojanje $apply$ morfizma, zahtjevamo pokazati da je i $B^A \wedge A \leq B$. Algebarskim računom dobivamo:
    \begin{align*}
      B^A \wedge A \leq B &= (\neg A \vee B) \wedge A\\
                        &= (\neg A \wedge A) \vee (B \wedge A)\\
                        &= 0 \vee (B \vee A)\\
                        &= B \vee A \leq B \\
    \end{align*}
    Da bi smo pokazali egistenciju morfizma $\lambda$ moramo pokazati da vrijedi:
    \begin{center}
      Za $C \in Obj_{\category{Pos(B)}}$ ako vrijedi $C \wedge A \leq B$ da tada vrijedi $C \leq B^A$
    \end{center}
    Pretpostavimo da za $C \in Obj_{\category{Pos(B)}}$ vrijedi $C \wedge A \leq B$. Tada imamo
    \begin{align*}
      C &= C \wedge 1\\
        &= C \wedge (A \vee \neg A)\\
        &= (C \wedge A) \vee (A \wedge \neg A)\\
        &\leq B \vee (A \wedge \neg A)\\
        &\leq B \vee (\neg A) = B^A \\
    \end{align*}

  \end{example}
  \newpage


\chapter{Kategorije redefinirane}
%%%%%%%%%%%%%%%%%%%%%%%%%%%%%%%%%%%%%%%%%%%%%
%
\section{Kategorije redefinirane}

\begin{definition}
  Uređeni par $(A, V)$, gdje je $A$ proizvoljan skup, a $V \subseteq A \times A$ nazivamo usmjereni graf. Za $f \in V$ pišemo još i $f : A \implies B$ ili $A \overset{f}{\implies} B$.
\end{definition}

\begin{definition}
  \emph{Dedukcijski graf} je usmjereni graf $\alpha = (A_0, A_1)$ za koji vrijedi:
  \begin{enumerate}
    \item Ako $f : A \implies B$, $g: B \implies C$ tada postoji $g \circ f : A \implies C$. $g \circ f$ zovemo kompozicijom $f$ i $g$.
    \item Za svaki $A \in A_0$ postoji $id^A : A \implies A$.
  \end{enumerate}

  Elemente skupa $A_0$ nazivamo \emph{objekti}, a elemente skupa $A_a$ \emph{strelice}. Posebno, strelicu $id^A$ nazivamo \emph{strelica identitete}. Za strelicu $f : A \implies B$ kažemo da je $A$ \emph{domena}, a $B$ \emph{kodomena} strelice f.
\end{definition}

\begin{definition}
  Kategorija je dedukcijski graf $\alpha$ takav da:
  \begin{enumerate}
    \item Za svaku strelicu $F : A \implies B$ vrijedi $f \circ id_A = f$, $id_B \circ f = f$
    \item Za sve strelice $f : A \implies B$, $g : B \implies C$, $h : C \implies D$ vrijedi: $h \circ (g \circ f) = (h \circ g) \circ f$.
  \end{enumerate}
\end{definition}


\begin{definition}
  \emph{Tci-dedukcijski graf} (truth, conjuction, implication) ili \emph{pozitivan dedukcijski graph} je uređena trojka $(A_0, A_1, \top)$, gdje je $(A_0, A_1)$ dedukcijski graf, a $\top \in A_0$, koji je zatvoren na binarne operacije $\land$ i $\to$, te vrijedi:
    \begin{enumerate}
      \item $tr^A : A \implies \top$ 
      \item $\pi^{A, B}_0 : A \land B \implies A$, $\pi^{A, B}_1 : A \land B \implies B$ i ako $f : C \implies A$, $g : C \implies B$ tada $<f, g> : C \implies A \land B$.
      \item $ev^{A, B} : (A \implies B) \land A \implies B$ i ako $h : C \land B \implies A$ then $cur(h): C \implies B \implies A$
    \end{enumerate}
\end{definition}

\begin{definition}
  Za dva objekta $A, B$ u kategoriji $alpha = (A_0, A_1)$ kazemo da su izomorfni, ako postoje strelice: $f : A \implies B$, $g : B -> A$ takve da $g \circ f = id^A$, $f \circ g = id^B$. Pisemo $A znak B$ i kazemo da su $A i B$ \emph{izomorfni}
\end{definition}

\begin{definition}
  Definicija CCC
\end{definition}

\begin{lema}
  CCC jednakosti

  \begin{equation*}
  <f, g> \circ h = <f \circ h, g \circ h>
  \end{equation*}
  \begin{equation*}
  ev<cur(f), g> = f<id, g> \\
  \end{equation*}
  \begin{equation*}
  cur(f) \circ g' = cur (f \circ <g' \circ \pi_0, \pi_1> \\
  \end{equation*}
\end{lema}

\begin{lema}
  bijekcija, $X(A, B)$ i $X(T, A \to B)$
\end{lema}



\chapter{Osvrt na lambda racun}
%%%%%%%%%%%%%%%%%%%%%%%%%%%%%%%%%%%%%%%%%%%%%
%
U ovom poglavlju dajemo pregled pojmova iz lamda racuna cije je razumjevanje nuzno kako bi se moglo definirati prosirenje iste, pomocu koje se uspostavlja bijekcija sa zatvorenim kartezijevim kategorijama (CCC). Definiramo pojmove kao stu su: jednostavne tipovi, termovi za lambda racun, supsitutcija, slobodne varijable itd.

\begin{definition}[Jednostavni tipovi]
  Neka je $T$ prebrojiv skup i $\to \, \in T \times T$.
    Ako za svaki $A, B \in T$, vrijedi $A \to B \in T$, tada kažemo da je $T_{\to}$ skup jednostavnih tipova (eng. simple types). Elemente skupa $T$ označavamo $P_0, P_1, P_2, ..$ i zovemo \emph{tipovske variable} (eng. type variables)
\end{definition}



\begin{definition}[Termovi jednostavno tipiziranog lambda računa $T_\to$.]
  Neka je $T_\to$ skup jednostavnih tipova. Tada možemo konstruirati termove jednostavnog tipiziranog lambda računa $T_\to$ na slijedeći način:
  \begin{itemize}
    \item za svaki $A \in T_\to$ postoji prebrojivo variabli tipa A: $x^A, y^A, z^A$, a svaka varijabla tipa $A$ je ujedno i term tipa $A$
    \item ako su $t^{A \to B}$ i $s^A$ termovi tipa $A \to B$ i $A$, tada $App(t^{A \to B}, S^A)$ je term tipa $B$
    \item ako je $t^B$ term tipa B i $x^A$ variabla tipa A, tada $(\lambda x^A . t^B)^{A \to B}$ je term tipa ${A \to B}$
  \end{itemize}
\end{definition}

\begin{definition}[Slobodne varijable]
  Skup $FV(t)$ slobodnih varijabli u $t$ je definiran rekurzivno:
  \begin{itemize}
    \item $FV(x : A) := x : A$
    \item $FV(ts) := FV(t) \cap FV(s)$
    \item $FV(\lambda x . t) := FV(t) \ {x}$
  \end{itemize}
  Za varijabu $x$ kazemo da je \emph{slobodna varijabla} u $t$, ako je $x \in FV(t)$. U suprotnom kazemo da je $x$ \emph{zauzeta varijabla}.
\end{definition}

\begin{definition}[Supstitucija]
  Rekurzivno definiramo \emph{supstituciju terma $s$ za varijablu $x$ u termu $t$}:
  \begin{itemize}
    \item $x[x/s] := s$
    \item $y[x/s] := y$ za $y \not\equiv x$
    \item $(t_1t_2)[x/s] := t_1[x/s]t_2[x/s]$
    \item $(\lambda x . t)[x / s] := \lambda x . t$
    \item $(\lambda y . t)[x / s] := \lambda y . t[x/s]$ za $y \not\equiv x$
  \end{itemize}
\end{definition}

\begin{primjer}
  Na primjeru $f := (\lambda x y . x y z) (x) (y)$ demonstriramo primjenu slobodnih varijabli i supsitutcije.
  Radimo supsitutciju terma $a$ za varijablu $x$ u termu $f$.
  \begin{align*}
    f[x/a]  &= (\lambda x y . x y z) (x) (y) [x/a] \\
            &= (\lambda x y . x y z)[x/a] (x)[x/a] (y) [x/a] \\ 
            &= (\lambda x y . x y z ) (a) (y) \\ 
  \end{align*}
  Gdje je svaka jednakost dobivena direktnom primjenom definicije na prijasnju. Analogno, slijedeci definiciju, dobivamo i skup $FV(f)$ slobodnih varijabli u $f$.
  \begin{align*}
    FV(t) &= FV((\lambda x y . x y z) (x) (y)) \\
          &= FV(\lambda x y . x y z) \cap FV(x) \cap FV(y) \\
          &= (FV(x y z) \setminus \{x, y\}) \cup \{x\} \cup \{y\} \\
          &= (FV(x) \cup FV(y) \cup FV(z) \setminus \{x, y\}) \cup \{x\} \cup \{y\} \\
          &= ((\{x\} \cup \{y\} \cup \{z\}) \setminus \{x, y\}) \cup \{x\} \cup \{y\} \\
          &= (\{x, y, z\}\setminus \{x, y\}) \cup \{x\} \cup \{y\} \\
          &= \{z\} \cup \{x\} \cup \{y\} \\
          &= \{x, y, z\}
  \end{align*}
\end{primjer}

Gornjim primjerom imamo dva sintakticki razlicita terma, $ (\lambda x y . x y z) (x) (y)$ i  $(\lambda x y . x y z) (a) (y)$, no njihovo semanticko znacenje zelimo da bude identicno. Primjetimo takoder da
\begin{center}
$FV((\lambda x y . x y z) (a) (y)) = \{a, y, z\} \neq \{x, y, z\} = FV(f)$.
\end{center}

sto se kosi sa intuitivnom intuicijom slobodnih varijabli. U tu srvhu definiramo konverzije, posebno $\alpha$, $\beta$ i $\eta$ konverzije, koje mozemo promatrati kao zamjenu varijable, aplikaciju funkcije i micanje apstrakcije. 

\begin{definition}[Konverzija]
  Neka je $T$ skup termova i $conv \in T \times T$ binarna relacija. Ako za $t, s \in T$ vrijedi $t\, conv\, s$ tada kazemo da $t$ konvertira u $s$, gdje $t$ zovemo $redeksom$ (eng. redex), $s$  ? (eng. conversum) od $t$. Zamjenu redexa sa conversumom zovemo \emph{konverzijom $t$ u $s$}. Pisemo $t \prec_1 s$ ako je $s$ dobiven od $t$ u jednoj konverziji. Relaciju $\prec$ definiramo kao tranzitivno zatvorenje relacije $\prec_1$, a relaciju $\preceq$ kao tranzitivno i refleksivno zatvorenje $\prec$. Analogno definiramo relacije $\succ$ i $\succeq$.
\end{definition}

Posebno, spominjemo tri vrste konverzija:
\begin{itemize}
  \item $\alpha-$konverzija: $\lambda x^A . x^A $ $cont_\alpha$ $\lambda y^A . y^A$
  \item $\beta-$konverzija: $(\lambda x^A . t^B) s^A$ $cont_\beta$ $t^B[x^A/s^A]$
  \item $\eta-$konverzija: $\lambda x^A . t x $ $cont_\eta$ $t$ $(x \not\in FV(t))$
\end{itemize} 

\begin{primjer}
  Neka je $g := (\lambda x y . x y z) (a) (y)$. Tada prema definiciji $\beta-$konverzije imamo
  \begin{equation*}
    (\lambda x y . x y z) (a) (y)\, cont_\beta \,(\lambda y. (x y z [x/a])) (y)  = (\lambda y.a y z) (y) cont_\beta \, a y z [y / y]  = a y z
  \end{equation*}
  Cime smo opravdali intuitivnu definiciju $\beta$-redukcije kao aplikaciju funkcije na sintakticnoj razini.
\end{primjer}

Sada, mozemo definirati i semanticku jednakost termova, tj. jednakost po kojoj ce termovi $f$ i $g$ biti jednaki.

\begin{definition}[Jednakost po konverziji]
  Neka je T skup termova i $=_{conv} \in T \times T$. Kažemo \emph{$t$ je jednak po konverziji $s$} i pišemo $t =_{conv} s$ ako ako postoji niz $t_o, t_1, t_2, .., t_n$,. takav da
  \begin{itemize}
    \item $t_0 \equiv t$
    \item $t_n \equiv s$
    \item $t_i \prec t_{i+1}$ ili $t_i \succ t_{i+1}$ za $i \in {0, 1, .., n}$
  \end{itemize}
\end{definition}

\begin{definition}[Normalna forma]
  Za term $t \in T$ kazemo da je u \emph{normalnoj formi} ako ne sadrzi redeks. $t$ ima normalnu formu ako vrijedi $t \succeq s$ i $s$ je u normalnoj formi.
\end{definition}

\begin{primjer}
  Primjer terma i normalne forme. Bitnost normalne forme?
\end{primjer}

\begin{definition}[Konfluentne relacije / Church-Rosser]
  Za relaciju $R$ kazemo da je \emph{konfluenta} (eng. confluent) ako za svaki $t_0, t_1, t_2 \in T$ za koji vrijedi $t_0 R t_1$ i $t_0 R T_2$, postoji $t_3 \in T$ takav da $t_1 R t_3$ i $t_2 R t_3$.
\end{definition}


\begin{teorem}
  Ako je $\preceq$ konfluentna relacija, tada
  \begin{center}
    $t = t'$ ako i samo ako postoji term $t''$ takav da $t \preceq t''$ i $t' \preceq t''$
  \end{center}
\end{teorem}

\begin{definition}[jednostavni tipizirani lambda racun]
  Neka je $\lambda_\to$ skup jednostavnih tipova. Definiramo aksiome $\lambda_\to$ racuna.
  \begin{enumerate*}
  
  \item \begin{center}$t \succeq t$\end{center}

  \item \begin{center}$(\lambda x^A . t ^B)s^A \succeq t^b [x^a / s^a]$\end{center}

  \item  \begin{prooftree}
      \AxiomC{$t \succeq s$}
      \UnaryInfC{$rt \succeq rs$}
    \end{prooftree}

  \item
    \begin{prooftree}
      \AxiomC{$t \succeq s$}
      \UnaryInfC{$tr \succeq sr$}
    \end{prooftree}

  \item
    \begin{prooftree}
      \AxiomC{$t \succeq s$}
      \UnaryInfC{$\lambda x . t \succeq \lambda x .s $}
    \end{prooftree}

  \item
    \begin{prooftree}
      \AxiomC{$t \succeq s$}
      \AxiomC{$s \succeq r$}
      \BinaryInfC{$t \succeq r$}
    \end{prooftree}


  \item
    \begin{prooftree}
      \AxiomC{$t \succeq s$}
      \UnaryInfC{$t = s$}
    \end{prooftree}

  \item
    \begin{prooftree}
      \AxiomC{$t = s$}
      \UnaryInfC{$s = t$}
    \end{prooftree}

  \item
    \begin{prooftree}
      \AxiomC{$t = s$}
      \AxiomC{$s = r$}
      \BinaryInfC{$t = r$}
    \end{prooftree}

  \end{enumerate*}
\end{definition}


\begin{definition}[Termovi tipizirane kombinatorne logike $CL_\to$]
  Induktivno definiramo termove kao i za $\lambda_\to$:
  \begin{itemize}
    \item za svaki $A \in T_\to$ postoji prebrojivo variabli tipa A: $x^A, y^A, z^A$, a svaka varijabla tipa $A$ je ujedno i term tipa $A$
    \item Za sve $A, B, C \in T$ postoje konstatni termovi (konstante)
      \begin{equation*}
        k^{A, B} \in A \to (B \to A)
      \end{equation*}
      \begin{equation*}
        s^{A, B, C} \in (A \to (B \to C)) \to ((A \to B) \to (A \to C))
      \end{equation*}
    \item ako je $t^{A \to B}$ i $s^A$ tada je $App(t^{A \to B}, s^a) ^B$ term tipa $B$
  \end{itemize}

\end{definition}

Napisati iskaze teorema i definirati sve sto se spominje u iskazu.
Zadnja 3-4 teorema - coherence theorems

conversum - kontraktum
koristiti redukciju
\begin{definition}[tipizirana kombinatorna logike $CL_\to$]
  Raspisati aksiome identicne kao i za $\lambda_\to$ nakon sto se prilagodi nekom citljivijem formatu (mozda u dva stupca?)
\end{definition}
\section{labda to ccc}

\begin{definition}
  calculus $\lambda \eta_{\to \top \bot}$
\end{definition}

\begin{definition}
  Kombinator, $\top$ kombinator
\end{definition}

\begin{definition}
  A, B konjukcija prop varijabli.
  $\xi_{AB}:A \implies B
\end{definition}

\begin{definition}
  $\Theta, \Theta', \overline{\Theta}, \overline{\Theta'}$, $\xi_{\Theta, \Theta}$
\end{definition}

\begin{lema}
  preslikavanje $r$, strelice iz ccc u tipizirane termove
\end{lema}

\begin{lema}
  preslikavanje $\sigma$, tipizirani termovi u strelice u ccc
\end{lema}

\begin{lema}
  preslikavanje $r$ cuva jednakost izmedu kombinatora
\end{lema}

\begin{lema}
  preslikavanje $\sigma$ invarijantno na $\beta$ redukciju
\end{lema}

\begin{lema}
  preslikavanje $\sigma$ cuda $\beta\eta$ jednakost
\end{lema}


\section{Coherence teorem}
\begin{definition}
  balansirana $\top \to$ formula
\end{definition}

\begin{teorem}
  Coherence teorem
\end{teorem}

\begin{definition}
  2-sequent
\end{definition}

\begin{teorem}
  Coherence teorem za 2-sequente
\end{teorem}

\begin{teorem}
  Redukcija na 2-sequent
\end{teorem}


%
%%%%%%%%%%%%%%%%%%%%%%%%%%%%%%%%%%%%%%%%%%%%%
%%%%%%%%%%%%%%%%%%%%%%%%%%%%%%%%%%%%%%%%%%%%%
%
\appendix
%
\renewcommand{\theequation}{A.\arabic{equation}}
%
%%%%%%%%%%%%%%%%%%%%%%%%%%%%%%%%%%%%%%%%%%%%%
% Bibliografija
%
\newpage
\addcontentsline{toc}{chapter}{Bibliografija}
\begin{thebibliography}{99}

\bibitem{Ferguson} T.S. Ferguson:
\textit{A Course in Large Sample Theory},
Chapman \& Hall/CRC, London, 1996.

\bibitem{Hajek} J. Hajek:
\textit{Some Extensions of Wald-Wolfowitz-Noether Theorem},
Ann. Math. Statistics, Vol. 32, 505-523, 1961.

\bibitem{Harding} E.F. Harding:
\textit{An Efficient, Minimal-storage Procedure for Calculating
the Mann-Whitney U, Generalized U and Similar Distributions},
Applied Statistics, Vol. 33, No. 1, 1-6. 1984.

\bibitem{Lehmann-Nonpar} E.L. Lehmann:
\textit{Nonparametrics: Statistical Methods Based on Ranks},
Revised First Edition, Springer, New York, 2006.

\bibitem{Mann-Whitney} H.B. Mann, D.R. Whitney:
\textit{On a Test of Whether One of Two Random
Variables is Stochastically Larger Than the Other},
Ann. Math. Statistics, Vol. 18, 50-60, 1947.

\bibitem{SarapaTV} N. Sarapa: \textit{Teorija vjerojatnosti},
\v{S}kolska knjiga, Zagreb, 2002.

\bibitem{Shiryaev-Prob} A.N. Shiryaev, \textit{Probability},
Second Edition, Springer, New York, 1996.

\bibitem{Sprent+Smeeton-Nonpar} P. Sprent, N.C. Smeeton,
\textit{Applied Nonparametric Statistical Methods},
Fourth Edition, Chapman \& Hall/CRC, Boca Raton, 2007.

\bibitem{Wilcoxon} F. Wilcoxon:
\textit{Individual Comparisons by Ranking Methods},
Biometrics, Vol. 1, 80-83, 1945.

\end{thebibliography}
%
%%%%%%%%%%%%%%%%%%%%%%%%%%%%%%%%%%%%%%%%%%%%%
%%%%%%%%%%%%%%%%%%%%%%%%%%%%%%%%%%%%%%%%%%%%%

\end{document}


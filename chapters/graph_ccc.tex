\section{Kategorije redefinirane}

\begin{definition}
  Uređeni par $(A, V)$, gdje je $A$ proizvoljan skup, a $V \subseteq A \times A$ nazivamo usmjereni graf. Za $f \in V$ pišemo još i $f : A \implies B$ ili $A \overset{f}{\implies} B$.
\end{definition}

\begin{definition}
  \emph{Dedukcijski graf} je usmjereni graf $\alpha = (A_0, A_1)$ za koji vrijedi:
  \begin{enumerate}
    \item Ako $f : A \implies B$, $g: B \implies C$ tada postoji $g \circ f : A \implies C$. $g \circ f$ zovemo kompozicijom $f$ i $g$.
    \item Za svaki $A \in A_0$ postoji $id^A : A \implies A$.
  \end{enumerate}

  Elemente skupa $A_0$ nazivamo \emph{objekti}, a elemente skupa $A_a$ \emph{strelice}. Posebno, strelicu $id^A$ nazivamo \emph{strelica identitete}. Za strelicu $f : A \implies B$ kažemo da je $A$ \emph{domena}, a $B$ \emph{kodomena} strelice f.
\end{definition}

\begin{definition}
  Kategorija je dedukcijski graf $\alpha$ takav da:
  \begin{enumerate}
    \item Za svaku strelicu $F : A \implies B$ vrijedi $f \circ id_A = f$, $id_B \circ f = f$
    \item Za sve strelice $f : A \implies B$, $g : B \implies C$, $h : C \implies D$ vrijedi: $h \circ (g \circ f) = (h \circ g) \circ f$.
  \end{enumerate}
\end{definition}


\begin{definition}
  \emph{Tci-dedukcijski graf} (truth, conjuction, implication) ili \emph{pozitivan dedukcijski graph} je uređena trojka $(A_0, A_1, \top)$, gdje je $(A_0, A_1)$ dedukcijski graf, a $\top \in A_0$, koji je zatvoren na binarne operacije $\land$ i $\to$, te vrijedi:
    \begin{enumerate}
      \item $tr^A : A \implies \top$ 
      \item $\pi^{A, B}_0 : A \land B \implies A$, $\pi^{A, B}_1 : A \land B \implies B$ i ako $f : C \implies A$, $g : C \implies B$ tada $<f, g> : C \implies A \land B$.
      \item $ev^{A, B} : (A \implies B) \land A \implies B$ i ako $h : C \land B \implies A$ then $cur(h): C \implies B \implies A$
    \end{enumerate}
\end{definition}

\begin{definition}
  Za dva objekta $A, B$ u kategoriji $alpha = (A_0, A_1)$ kazemo da su izomorfni, ako postoje strelice: $f : A \implies B$, $g : B -> A$ takve da $g \circ f = id^A$, $f \circ g = id^B$. Pisemo $A znak B$ i kazemo da su $A i B$ \emph{izomorfni}
\end{definition}

\begin{definition}
  Definicija CCC
\end{definition}

\begin{lema}
  CCC jednakosti

  \begin{equation*}
  <f, g> \circ h = <f \circ h, g \circ h>
  \end{equation*}
  \begin{equation*}
  ev<cur(f), g> = f<id, g> \\
  \end{equation*}
  \begin{equation*}
  cur(f) \circ g' = cur (f \circ <g' \circ \pi_0, \pi_1> \\
  \end{equation*}
\end{lema}

\begin{lema}
  bijekcija, $X(A, B)$ i $X(T, A \to B)$
\end{lema}


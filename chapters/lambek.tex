U ovom poglavlju dajemo pregled pojmova iz lamda racuna cije je razumjevanje nuzno kako bi se moglo definirati prosirenje iste, pomocu koje se uspostavlja bijekcija sa zatvorenim kartezijevim kategorijama (CCC). Definiramo pojmove kao stu su: jednostavne tipovi, termovi za lambda racun, supsitutcija, slobodne varijable itd.

\begin{definition}[Jednostavni tipovi]
  Neka je $T$ prebrojiv skup i $\to \, \in T \times T$.
    Ako za svaki $A, B \in T$, vrijedi $A \to B \in T$, tada kažemo da je $T_{\to}$ skup jednostavnih tipova (eng. simple types). Elemente skupa $T$ označavamo $P_0, P_1, P_2, ..$ i zovemo \emph{tipovske variable} (eng. type variables)
\end{definition}



\begin{definition}[Termovi jednostavno tipiziranog lambda računa $T_\to$.]
  Neka je $T_\to$ skup jednostavnih tipova. Tada možemo konstruirati termove jednostavnog tipiziranog lambda računa $T_\to$ na slijedeći način:
  \begin{itemize}
    \item za svaki $A \in T_\to$ postoji prebrojivo variabli tipa A: $x^A, y^A, z^A$, a svaka varijabla tipa $A$ je ujedno i term tipa $A$
    \item ako su $t^{A \to B}$ i $s^A$ termovi tipa $A \to B$ i $A$, tada $App(t^{A \to B}, S^A)$ je term tipa $B$
    \item ako je $t^B$ term tipa B i $x^A$ variabla tipa A, tada $(\lambda x^A . t^B)^{A \to B}$ je term tipa ${A \to B}$
  \end{itemize}
\end{definition}

\begin{definition}[Slobodne varijable]
  Skup $FV(t)$ slobodnih varijabli u $t$ je definiran rekurzivno:
  \begin{itemize}
    \item $FV(x : A) := x : A$
    \item $FV(ts) := FV(t) \cap FV(s)$
    \item $FV(\lambda x . t) := FV(t) \ {x}$
  \end{itemize}
  Za varijabu $x$ kazemo da je \emph{slobodna varijabla} u $t$, ako je $x \in FV(t)$. U suprotnom kazemo da je $x$ \emph{zauzeta varijabla}.
\end{definition}

\begin{definition}[Supstitucija]
  Rekurzivno definiramo \emph{supstituciju terma $s$ za varijablu $x$ u termu $t$}:
  \begin{itemize}
    \item $x[x/s] := s$
    \item $y[x/s] := y$ za $y \not\equiv x$
    \item $(t_1t_2)[x/s] := t_1[x/s]t_2[x/s]$
    \item $(\lambda x . t)[x / s] := \lambda x . t$
    \item $(\lambda y . t)[x / s] := \lambda y . t[x/s]$ za $y \not\equiv x$
  \end{itemize}
\end{definition}

\begin{primjer}
  Na primjeru $f := (\lambda x y . x y z) (x) (y)$ demonstriramo primjenu slobodnih varijabli i supsitutcije.
  Radimo supsitutciju terma $a$ za varijablu $x$ u termu $f$.
  \begin{align*}
    f[x/a]  &= (\lambda x y . x y z) (x) (y) [x/a] \\
            &= (\lambda x y . x y z)[x/a] (x)[x/a] (y) [x/a] \\ 
            &= (\lambda x y . x y z ) (a) (y) \\ 
  \end{align*}
  Gdje je svaka jednakost dobivena direktnom primjenom definicije na prijasnju. Analogno, slijedeci definiciju, dobivamo i skup $FV(f)$ slobodnih varijabli u $f$.
  \begin{align*}
    FV(t) &= FV((\lambda x y . x y z) (x) (y)) \\
          &= FV(\lambda x y . x y z) \cap FV(x) \cap FV(y) \\
          &= (FV(x y z) \setminus \{x, y\}) \cup \{x\} \cup \{y\} \\
          &= (FV(x) \cup FV(y) \cup FV(z) \setminus \{x, y\}) \cup \{x\} \cup \{y\} \\
          &= ((\{x\} \cup \{y\} \cup \{z\}) \setminus \{x, y\}) \cup \{x\} \cup \{y\} \\
          &= (\{x, y, z\}\setminus \{x, y\}) \cup \{x\} \cup \{y\} \\
          &= \{z\} \cup \{x\} \cup \{y\} \\
          &= \{x, y, z\}
  \end{align*}
\end{primjer}

Gornjim primjerom imamo dva sintakticki razlicita terma, $ (\lambda x y . x y z) (x) (y)$ i  $(\lambda x y . x y z) (a) (y)$, no njihovo semanticko znacenje zelimo da bude identicno. Primjetimo takoder da
\begin{center}
$FV((\lambda x y . x y z) (a) (y)) = \{a, y, z\} \neq \{x, y, z\} = FV(f)$.
\end{center}

sto se kosi sa intuitivnom intuicijom slobodnih varijabli. U tu srvhu definiramo konverzije, posebno $\alpha$, $\beta$ i $\eta$ konverzije, koje mozemo promatrati kao zamjenu varijable, aplikaciju funkcije i micanje apstrakcije. 

\begin{definition}[Konverzija]
  Neka je $T$ skup termova i $conv \in T \times T$ binarna relacija. Ako za $t, s \in T$ vrijedi $t\, conv\, s$ tada kazemo da $t$ konvertira u $s$, gdje $t$ zovemo $redeksom$ (eng. redex), $s$  ? (eng. conversum) od $t$. Zamjenu redexa sa conversumom zovemo \emph{konverzijom $t$ u $s$}. Pisemo $t \prec_1 s$ ako je $s$ dobiven od $t$ u jednoj konverziji. Relaciju $\prec$ definiramo kao tranzitivno zatvorenje relacije $\prec_1$, a relaciju $\preceq$ kao tranzitivno i refleksivno zatvorenje $\prec$. Analogno definiramo relacije $\succ$ i $\succeq$.
\end{definition}

Posebno, spominjemo tri vrste konverzija:
\begin{itemize}
  \item $\alpha-$konverzija: $\lambda x^A . x^A $ $cont_\alpha$ $\lambda y^A . y^A$
  \item $\beta-$konverzija: $(\lambda x^A . t^B) s^A$ $cont_\beta$ $t^B[x^A/s^A]$
  \item $\eta-$konverzija: $\lambda x^A . t x $ $cont_\eta$ $t$ $(x \not\in FV(t))$
\end{itemize} 

\begin{primjer}
  Neka je $g := (\lambda x y . x y z) (a) (y)$. Tada prema definiciji $\beta-$konverzije imamo
  \begin{equation*}
    (\lambda x y . x y z) (a) (y)\, cont_\beta \,(\lambda y. (x y z [x/a])) (y)  = (\lambda y.a y z) (y) cont_\beta \, a y z [y / y]  = a y z
  \end{equation*}
  Cime smo opravdali intuitivnu definiciju $\beta$-redukcije kao aplikaciju funkcije na sintakticnoj razini.
\end{primjer}

Sada, mozemo definirati i semanticku jednakost termova, tj. jednakost po kojoj ce termovi $f$ i $g$ biti jednaki.

\begin{definition}[Jednakost po konverziji]
  Neka je T skup termova i $=_{conv} \in T \times T$. Kažemo \emph{$t$ je jednak po konverziji $s$} i pišemo $t =_{conv} s$ ako ako postoji niz $t_o, t_1, t_2, .., t_n$,. takav da
  \begin{itemize}
    \item $t_0 \equiv t$
    \item $t_n \equiv s$
    \item $t_i \prec t_{i+1}$ ili $t_i \succ t_{i+1}$ za $i \in {0, 1, .., n}$
  \end{itemize}
\end{definition}

\begin{definition}[Normalna forma]
  Za term $t \in T$ kazemo da je u \emph{normalnoj formi} ako ne sadrzi redeks. $t$ ima normalnu formu ako vrijedi $t \succeq s$ i $s$ je u normalnoj formi.
\end{definition}

\begin{primjer}
  Primjer terma i normalne forme. Bitnost normalne forme?
\end{primjer}

\begin{definition}[Konfluentne relacije / Church-Rosser]
  Za relaciju $R$ kazemo da je \emph{konfluenta} (eng. confluent) ako za svaki $t_0, t_1, t_2 \in T$ za koji vrijedi $t_0 R t_1$ i $t_0 R T_2$, postoji $t_3 \in T$ takav da $t_1 R t_3$ i $t_2 R t_3$.
\end{definition}


\begin{teorem}
  Ako je $\preceq$ konfluentna relacija, tada
  \begin{center}
    $t = t'$ ako i samo ako postoji term $t''$ takav da $t \preceq t''$ i $t' \preceq t''$
  \end{center}
\end{teorem}

\begin{definition}[jednostavni tipizirani lambda racun]
  Neka je $\lambda_\to$ skup jednostavnih tipova. Definiramo aksiome $\lambda_\to$ racuna.
  \begin{enumerate*}
  
  \item \begin{center}$t \succeq t$\end{center}

  \item \begin{center}$(\lambda x^A . t ^B)s^A \succeq t^b [x^a / s^a]$\end{center}

  \item  \begin{prooftree}
      \AxiomC{$t \succeq s$}
      \UnaryInfC{$rt \succeq rs$}
    \end{prooftree}

  \item
    \begin{prooftree}
      \AxiomC{$t \succeq s$}
      \UnaryInfC{$tr \succeq sr$}
    \end{prooftree}

  \item
    \begin{prooftree}
      \AxiomC{$t \succeq s$}
      \UnaryInfC{$\lambda x . t \succeq \lambda x .s $}
    \end{prooftree}

  \item
    \begin{prooftree}
      \AxiomC{$t \succeq s$}
      \AxiomC{$s \succeq r$}
      \BinaryInfC{$t \succeq r$}
    \end{prooftree}


  \item
    \begin{prooftree}
      \AxiomC{$t \succeq s$}
      \UnaryInfC{$t = s$}
    \end{prooftree}

  \item
    \begin{prooftree}
      \AxiomC{$t = s$}
      \UnaryInfC{$s = t$}
    \end{prooftree}

  \item
    \begin{prooftree}
      \AxiomC{$t = s$}
      \AxiomC{$s = r$}
      \BinaryInfC{$t = r$}
    \end{prooftree}

  \end{enumerate*}
\end{definition}


\begin{definition}[Termovi tipizirane kombinatorne logike $CL_\to$]
  Induktivno definiramo termove kao i za $\lambda_\to$:
  \begin{itemize}
    \item za svaki $A \in T_\to$ postoji prebrojivo variabli tipa A: $x^A, y^A, z^A$, a svaka varijabla tipa $A$ je ujedno i term tipa $A$
    \item Za sve $A, B, C \in T$ postoje konstatni termovi (konstante)
      \begin{equation*}
        k^{A, B} \in A \to (B \to A)
      \end{equation*}
      \begin{equation*}
        s^{A, B, C} \in (A \to (B \to C)) \to ((A \to B) \to (A \to C))
      \end{equation*}
    \item ako je $t^{A \to B}$ i $s^A$ tada je $App(t^{A \to B}, s^a) ^B$ term tipa $B$
  \end{itemize}

\end{definition}

Napisati iskaze teorema i definirati sve sto se spominje u iskazu.
Zadnja 3-4 teorema - coherence theorems

conversum - kontraktum
koristiti redukciju
\begin{definition}[tipizirana kombinatorna logike $CL_\to$]
  Raspisati aksiome identicne kao i za $\lambda_\to$ nakon sto se prilagodi nekom citljivijem formatu (mozda u dva stupca?)
\end{definition}
\section{labda to ccc}

\begin{definition}
  calculus $\lambda \eta_{\to \top \bot}$
\end{definition}

\begin{definition}
  Kombinator, $\top$ kombinator
\end{definition}

\begin{definition}
  A, B konjukcija prop varijabli.
  $\xi_{AB}:A \implies B
\end{definition}

\begin{definition}
  $\Theta, \Theta', \overline{\Theta}, \overline{\Theta'}$, $\xi_{\Theta, \Theta}$
\end{definition}

\begin{lema}
  preslikavanje $r$, strelice iz ccc u tipizirane termove
\end{lema}

\begin{lema}
  preslikavanje $\sigma$, tipizirani termovi u strelice u ccc
\end{lema}

\begin{lema}
  preslikavanje $r$ cuva jednakost izmedu kombinatora
\end{lema}

\begin{lema}
  preslikavanje $\sigma$ invarijantno na $\beta$ redukciju
\end{lema}

\begin{lema}
  preslikavanje $\sigma$ cuda $\beta\eta$ jednakost
\end{lema}


\section{Coherence teorem}
\begin{definition}
  balansirana $\top \to$ formula
\end{definition}

\begin{teorem}
  Coherence teorem
\end{teorem}

\begin{definition}
  2-sequent
\end{definition}

\begin{teorem}
  Coherence teorem za 2-sequente
\end{teorem}

\begin{teorem}
  Redukcija na 2-sequent
\end{teorem}
